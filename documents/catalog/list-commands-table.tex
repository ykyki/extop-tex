\documentclass[uplatex, dvipdfmx, 12pt, crop=false]{standalone}
\usepackage{import}
% \import{./}{common-preamble.sty}
\import{../config/}{global-commands.sty}


\usepackage[
	paperwidth=330truemm,
	paperheight=600truemm,
	top=10truemm,
	bottom=15truemm,
	marginparwidth=15truemm,
	marginparsep=5truemm
]{geometry}

\def\texorpdfstring#1#2{#1}% hyperref依存なのでここでは暫定的に無効化

\begin{document}

\newcommand{\tablesubtitle}[1]{\multicolumn{2}{c}{\textbf{#1}}}

\begin{table}[htb]
	\centering
	\renewcommand{\arraystretch}{1.2}
	% \setlength{\tabcolsep}{1em}
	% \setlength{\doublerulesep}{10pt}
	% \setlength{\arrayrulewidth}{1pt}
	\caption{コマンド一覧}
	\begin{tabular}{ll@{\qquad}l}
		\hline
		\hline
		\tablesubtitle{基礎的なテキスト記号} \\
		(1)\rimp(2)  & \verb|(1)\rimp(2)|  & 推論を表す右向き矢印 \\
		(1)\limp(2)  & \verb|(1)\limp(2)|  & 推論を表す左向き矢印 \\
		(1)\lrimp(2) & \verb|(1)\lrimp(2)| & 同値を表す両向き矢印 \\
		\hline

		\tablesubtitle{基礎的な数式記号} \\
		$\N$                                      & \verb|\N|                                                                        & 自然数全体の集合                   \\
		$\Z$                                      & \verb|\Z|                                                                        & 整数全体の集合                     \\
		$\Q$                                      & \verb|\Q|                                                                        & 有理数全体の集合                   \\
		$\R$                                      & \verb|\R|                                                                        & 実数全体の集合                     \\
		$\C$                                      & \verb|\C|                                                                        & 複素数全体の集合                   \\
		$\I$                                      & \verb|\I|                                                                        & 単位区間                           \\
		$x \defeq f(a)$                           & \verb|x \defeq f(a)|                                                             & 定義を表す等号                     \\
		$x \in \complement A \defarw x \not\in A$ & \verb|x \in \complement A \defarw x \not\in A|                                   & 定義を表す矢印                     \\
		$\lrparen{f, g, h, \ldots}$               & \verb|\lrparen{f, g, h, \ldots}|                                                 & parenthesis                        \\
		$\lrparenbig{f, g, h, \ldots}$            & \verb|\lrparenbig{f, g, h, \ldots}|                                              & big parenthesis                    \\
		$\lrbrace{f, g, h, \ldots}$               & \verb|\lrbrace{f, g, h, \ldots}|                                                 & brace                              \\
		$\lrbrack{f, g, h, \ldots}$               & \verb|\lrbrack{f, g, h, \ldots}|                                                 & bracket                            \\
		$\lrangle{f, g, h, \ldots}$               & \verb|\lrangle{f, g, h, \ldots}|                                                 & angle bracket                      \\
		$\forallparen{\epsilon > 0}$              & \verb|\forallparen{\epsilon > 0}|                                                & 全称量化子を括る括弧               \\
		$\existsparen{\delta > 0}$                & \verb|\existsparen{\delta > 0}|                                                  & 存在量化子を括る括弧               \\
		$\formulaparen{x < y}$                    & \verb|\formulaparen{x < y}|                                                      & 論理式を括る括弧                   \\
		$\setcomp{x \in \R}{f(x) < 0}$            & \verb|\setcomp{x \in \R}{f(x) < 0}|                                              & 内包表記                           \\
		$\setfamily{X_i}{i \in I}$                & \verb|\setfamily{X_i}{i \in I}|                                                  & 添字付けられた集合系               \\
		$\cardinality{A}$                         & \verb|\cardinality{A}|                                                           & 集合$A$の濃度                      \\
		$\pow A, \pow(A)$                         & \verb|\pow A, \pow(A)|                                                           & 集合$A$の冪集合                    \\
		$\mappt{f}{a_U^N}, f(a_U^N)$              & \verb|\mappt{f}{a_U^N}, f(a_U^N)|                                                & 写像$f$により点$a_U^N$に対応する値 \\
                                                  & \multicolumn{2}{@{\quad}|l}{括弧の中身が複雑になりそうだったら前者を使う方針で} \\
		$\mapinvpt{h}{z}$                         & \verb|\mapinvpt{h}{z}|                                                           & 写像$h$による点$z$の逆像           \\
		$\mapset{f}{A}$                           & \verb|\mapset{f}{A}|                                                             & 写像$f$による集合$A$の順像         \\
		$\mapinvset{h}{K}$                        & \verb|\mapinvset{h}{K}|                                                          & 写像$h$による集合$K$の逆像         \\
		$\abs{a}$                                 & \verb|\abs{a}|                                                                   & 値$a$の絶対値                      \\
		\hline

		\tablesubtitle{位相空間に関する演算} \\
		$\topint_X A$               & \verb|\topint_X A|               & 空間$X$における部分集合$A$の内部                          \\
		$\topint A$                 & \verb|\topint A|                 & 部分集合$A$の内部                                         \\
		$\topcl_X A$                & \verb|\topcl_X A|                & 空間$X$における部分集合$A$の閉包                          \\
		$\topcl A$                  & \verb|\topcl A|                  & 部分集合$A$の閉包                                         \\
		$\topbar{A}$                & \verb|\topbar{A}|                & 部分集合$A$の閉包                                         \\
		$\topder_X A$               & \verb|\topder_X A|               & 空間$X$における部分集合$A$の導集合                        \\
		$\topder A$                 & \verb|\topder A|                 & 部分集合$A$の導集合                                       \\
		$A^\topd$                   & \verb|A^\topd|                   & 部分集合$A$の導集合                                       \\
		$\Star{A}{\mathscr{U}}$     & \verb|\Star{A}{\mathscr{U}}|     & 部分集合族$\mathscr{U}$に関する部分集合$A$の星型集合      \\
		$\Star[^n]{A}{\mathscr{U}}$ & \verb|\Star[^n]{A}{\mathscr{U}}| & 部分集合族$\mathscr{U}$に関する部分集合$A$の$n$階星型集合 \\
		\hline

		\tablesubtitle{位相空間に関するテキスト記号} \\
		\topT{1}, \topT{2}    & \verb|\topT{1}, \topT{2}|    & 分離公理を表す記号    \\
		\Gdelta 集合          & \verb|\Gdelta 集合|          & \Gdelta 集合          \\
		\Fsigma 集合          & \verb|\Fsigma 集合|          & \Fsigma 集合          \\
		\hline

		\tablesubtitle{位相空間に関する数式記号} \\
		$\topopens[X]$                          & \verb|\topopens[X]|                          & 空間$X$の開集合族                                       \\
		$\topopens$                             & \verb|\topopens|                             & 開集合族                                                \\
		$\topbasis$                             & \verb|\topbasis|                             & 開基                                                    \\
		$\topnbd[X]{x}$                         & \verb|\topnbd[X]{x}|                         & 空間$X$における点$x$の近傍族                            \\
		$\topnbd{x}$                            & \verb|\topnbd{x}|                            & 点$x$の近傍族                                           \\
		$\topball{x}{r}$                        & \verb|\topball{x}{r}|                        & 点$x$を中心する半径$r$の開球                            \\
		$\topdensity{X}$                        & \verb|\topdensity{X}|                        & 位相空間$X$の density                                   \\
		$\topweight{X}$                         & \verb|\topweight{X}|                         & 位相空間$X$のウェイト                                   \\
		$\topnetworkweight{X}$                  & \verb|\topnetworkweight{X}|                  & 位相空間$X$の ネットワーク濃度                          \\
		$\topcharacter{X}$                      & \verb|\topcharacter{X}|                      & 位相空間$X$の character                                 \\
		$\mathscr{U} \refines \mathscr{V}$      & \verb|\mathscr{U} \refines \mathscr{V}|      & 集合族$\mathscr{U}$が$\mathscr{V}$を細分する            \\
		$\mathscr{U} \deltarefines \mathscr{V}$ & \verb|\mathscr{U} \deltarefines \mathscr{V}| & 集合族$\mathscr{U}$が$\mathscr{V}$を$\Delta$-細分する \\
		$\mathscr{U} \starrefines \mathscr{V}$  & \verb|\mathscr{U} \starrefines \mathscr{V}|  & 集合族$\mathscr{U}$が$\mathscr{V}$を星型細分する        \\
		\hline

		\tablesubtitle{人名} \\
		\Alexandroff & \verb|\Alexandroff| &  \\
		\Baire       & \verb|\Baire|       &  \\
		\Cech        & \verb|\Cech|        &  \\
		\Euclid      & \verb|\Euclid|      &  \\
		\Frechet     & \verb|\Frechet|     &  \\
		\Hausdorff   & \verb|\Hausdorff|   &  \\
		\Kolmogorov  & \verb|\Kolmogorov|  &  \\
		\Lindelof    & \verb|\Lindelof|    &  \\
		\Moore       & \verb|\Moore|       &  \\
		\Sierpinski  & \verb|\Sierpinski|  &  \\
		\Sorgenfrey  & \verb|\Sorgenfrey|  &  \\
		\Stone       & \verb|\Stone|       &  \\
		\Tietze      & \verb|\Tietze|      &  \\
		\Tychonoff   & \verb|\Tychonoff|   &  \\
		\Urysohn     & \verb|\Urysohn|     &  \\
		\hline

		\tablesubtitle{その他} \\
		\ykyki & \verb|\ykyki| & 著者 \\
		\chen  & \verb|\chen|  & 著者 \\

		\hline
		\hline
	\end{tabular}
\end{table}

\end{document}
