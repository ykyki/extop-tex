\documentclass[uplatex, dvipdfmx, a4paper, 12pt, class=jsbook, crop=false]{standalone}
\usepackage{import}
\import{../}{common-preamble.sty}

\begin{document}
\section{弧状連結空間}
\label{sec:path-connected-spaces}

\newcommand{\loclabel}[1]{\label{LocalLabel-\thepart-\thechapter-\thesection:#1}}

\begin{definition}
	位相空間$ X $が\indexjj{こじょうれんけつ}{弧状連結}{path connected}であるとは,
	$ X $が空でなく$ X $の任意の2点$ x, y $に対し, この2点を結ぶ$ X $内の道が存在することである. 
\end{definition}

\begin{proposition}
	\loclabel{prop:Continuous maps preserve pathconnectedness}
	弧状連結性は連続写像によって保たれる.
\end{proposition}

\begin{proposition}
	\loclabel{prop:Ctd+LocPathCtd>PathCtd}
	位相空間$ X $が連結かつ局所弧状連結ならば弧状連結である.
\end{proposition}

\end{document}
