\documentclass[uplatex, dvipdfmx, a4paper, 12pt, class=jsbook, crop=false]{standalone}
\usepackage{import}
\import{../}{common-preamble.sty}

\begin{document}
\section{性質の保存}
\label{sec:preservation-of-properties}

\begin{definition}
	位相空間$ X $の性質$ \mathcal{P} $が\indexj{れんぞくしゃぞうでたもたれる}{連続写像で保たれる}とは,
	$ X $上の任意の連続全射$ \morph{f}{X}{Y} $について,
	位相空間$ Y $も性質$ \mathcal{P} $を満たすことをいう.
\end{definition}

閉写像や開写像, 商写像, 完全写像などの場合についても,
性質$ \mathcal{P} $が閉(開, 商, 完全)写像で保たれるという.

\begin{definition}
	位相空間の性質$ \mathcal{P} $が\indexj{ぶぶんくうかんにいでんする}{部分空間に遺伝する}とは,
	性質$ \mathcal{P} $を満たす任意の位相空間$ X $について,
	その任意の部分空間が性質$ \mathcal{P} $を満たすことをいう.
\end{definition}

\begin{definition}
	位相空間の性質$ \mathcal{P} $が\indexj{せきでたもたれる}{積で保たれる}とは,
	性質$ \mathcal{P} $を満たす位相空間からなる任意の族
	$ \setfamily{X_\lambda}{\lambda \in \Lambda} $について,
	その積空間$ \prod_{\lambda \in \Lambda}  X_\lambda $
	が性質$ \mathcal{P} $を満たすことをいう.
	特に, 有限個や可算個の場合について成り立つ場合は,
	有限個(可算個)の積で保たれるという.
\end{definition}

\end{document}
