\documentclass[uplatex, dvipdfmx, a4paper, 12pt, class=jsbook, crop=false]{standalone}
\usepackage{import}
\import{../}{common-preamble.sty}

\begin{document}
\section{k-空間}
\label{sec:k-spaces}

\begin{source}
	\cite[Section 7.9]{Dieck2008AT},
	\cite{Strickland2009CGWH},
	\cite{Rezk2017CGS},
	\cite{Rezk0000K}.
	% \cite{KOPSW2019CCTK}.
\end{source}

コンパクト\Hausdorff 空間のことを\textbf{ch-空間}と略記することにする.
また, 位相空間$X$に対し, ch-空間$K$から$X$への連続写像を$X$の\textbf{テストマップ}と呼ぶことにする.

\begin{definition}
	位相空間$X$の部分集合$A$が\indexjj{k-へいしゅうごう}{k-閉集合}{k-closed set}であるとは,
	$X$への任意のテストマップ$\morph{t}{K}{X}$に対し,
	逆像$\mapinvset{t}{A}$が$K$の閉集合になることである.
	同様にして\indexjj{k-かい集合}{k-開集合}{k-open set}も定義される.
\end{definition}

定義より, 閉集合は常にk-閉集合でもある.
開集合は常にk-開集合でもある
位相空間$X$のk-開集合全体を集めた部分集合族は集合$X$上の新たな位相を定める.
この位相空間を$kX$と表すことにする.

\begin{definition}
	位相空間$X$が\indexjj{k-くうかん}{k-空間}{k-space}であるとは,
	任意のk-閉集合が閉集合になることである.
\end{definition}

\begin{proposition}
	k-空間$X$の商空間はk-空間である.
\end{proposition}

\begin{proof}
	ほぼ straightforward.
	\WIP.
\end{proof}

\begin{proposition}
	k-空間$X$の閉集合$F$はk-空間である.
\end{proposition}

\begin{proof}
	ほぼ straightforward.
	\WIP.
\end{proof}

\begin{proposition}
	k-空間$X$の開集合$G$はk-空間である.
\end{proposition}

\begin{proof}
	\WIP.
\end{proof}

\begin{proposition}
	位相空間$X$について以下の条件は同値である:
	\begin{enumerate}
		\item $X$はk-空間である.
		\item $X$はある局所コンパクト\Hausdorff 空間の商空間である.
	\end{enumerate}
\end{proposition}

\begin{proof}
	\WIP.
\end{proof}

\begin{proposition}[{\cite[Proposition~7.9.7]{Dieck2008AT}}]
	\label{043c4f}
	$X$を弱\Hausdorff 位相空間とする.
	部分集合$A$について以下の条件は同値である:
	\begin{enumerate}
		\item $A$はk-閉集合である.
		\item $X$のch-部分集合全体が定める弱位相について, $A$は閉集合である.
	\end{enumerate}
\end{proposition}

\begin{proof}
	\WIP.
\end{proof}

\begin{corollary}
	$X$を弱\Hausdorff 位相空間とする.
	このとき以下の条件は同値である:
	\begin{enumerate}
		\item $X$はk-空間である.
		\item $X$はch-部分集合全体について弱位相をもつ.
	\end{enumerate}
\end{corollary}

\begin{proof}
	\cref{043c4f}より.
\end{proof}

\begin{proposition}
	列型\Hausdorff 空間はk-空間である.
\end{proposition}

\begin{proposition}
	\WIP.
\end{proposition}

\end{document}
