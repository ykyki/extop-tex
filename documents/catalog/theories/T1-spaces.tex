\documentclass[uplatex, dvipdfmx, a4paper, 12pt, class=jsbook, crop=false]{standalone}
\usepackage{import}
\import{../}{common-preamble.sty}

\begin{document}
\section{\topT{1}空間}
\label{sec:T1-spaces}

\begin{source}
	本節は主に\cite[Chapter 16]{Schechter1997HAF}を参考にしている.
\end{source}

\begin{definition}
	位相空間$ X $が\indexj{T1 くうかん}{\topT{1}空間}であるとは, \topT{0}かつ対称空間であることである.
\end{definition}

\begin{proposition}
	\label{T100001}
	位相空間$ X $について以下の条件は同値である:
	\begin{enumerate}
		\item $ X $が\topT{1}空間である.
		\item $ X $の任意の1点集合が閉である.
		\item $ X $の任意の点$ x $に対し次が成り立つ: $ x $と異なる任意の点$ y \in X $に対して開集合$ G $が存在して$ x \in G $かつ$ y \not\in G $となる.
	\end{enumerate}
	\qed
\end{proposition}

\begin{proposition}
	\label{T100002}
	\topT{0}空間は積で保たれる.
\end{proposition}

\end{document}
