\documentclass[uplatex, dvipdfmx, a4paper, 12pt, class=jsbook, crop=false]{standalone}
\usepackage{import}
\import{../}{common-preamble.sty}

\begin{document}
\section{擬コンパクト空間}
\label{sec:pseudo-compact-spaces}

\begin{definition}
	位相空間$ X $が\indexjj{ぎこんぱくと}{擬コンパクト}{pseudo compact}であるとは, $ X $上で定義される任意の実連続関数が有界となることである.
\end{definition}

擬コンパクト空間の部分空間も擬コンパクトになるとは限らない. 例えば, 通常の距離空間$ \Q $の一点コンパクト化$ \Q^\star $(\ref{example:Q_star})はコンパクトなので擬コンパクトでもあるが, 部分空間$ \Q $は擬コンパクトでない. ただし, 擬コンパクトな \topT{4} 空間の閉部分空間は擬コンパクトである. これは, Tietze の拡張定理からわかる. また, 擬コンパクト空間の連続像も擬コンパクトである.

\begin{proposition}
	\topT{3.5} 空間$ X $において次は同値である.
	\begin{enumerate}
		\item 擬コンパクトである.
		\item 任意の局所有限な開集合族は有限である.
		\item $ X $の空でない開集合からなる任意の下降列$ U_0 \supset U_1 \supset \cdots $に対して, $ \bigcap_{n \in \N} \topcl U_n \neq \emptyset $が成り立つ.
		\item 有限交叉性をもつ任意の可算な開集合族$ \{U_n\}_{n \in \N} $に対して,  $ \bigcap_{n \in \N} \topcl U_n \neq \emptyset $が成り立つ.
	\end{enumerate}
\end{proposition}

\begin{proof}
	(1) $ \Longrightarrow $ (2) : $ X $を擬コンパクトな \topT{3.5} 空間, $ \mathscr{U} $を非空開集合からなる局所有限な族とする. $ \mathscr{U} $が有限でないと仮定して矛盾を導く. ここで, 可算無限部分集合$ \{V_n\}_{n \in \N} \subset \mathscr{U} $をとる. 任意の$ n \in \N $に対して, $ x_n \in V_n $なる$ x_n \in X $をとると, $ X $が\topT{3.5}であることから, 各$ n \in \N $に対して$ f_n(x_n) = n, f[X \setminus V_n] \subset \{0\} $を満たす連続関数$ f_n \colon X \to \R $が存在する. このとき, $ f \colon X \to \R $を$ f \semicolon x \mapsto  \sum_{n \in \N} |f_n(x)| $と定めることができ, $ \{V_n\}_{n \in \N} $が有限交叉であることから$ f $は発散せず, かつ連続関数であることがわかる. 定義から$ f $は有界でないので, $ X $が擬コンパクトであることに矛盾する.

	(2) $ \Longrightarrow $ (3) : 任意の$ n \in \N $について$ U_n \supsetneq U_{n+1} $の場合について示せばよい. このとき, $ U_n $は有限でないので仮定より局所有限でない. したがって, ある$ x \in X $であって任意の近傍が$ \{U_n\}_{n \in \N} $の無限個の元と交わるものが存在するので, $ \{U_n\}_{n \in \N} $が下降列であることに注意すれば$ x \in \bigcap_{n \in \N} \topcl U_n $である.

	(3) $ \Longrightarrow $ (4) : $ \{U_n\}_{n \in \N} $を有限交叉性をもつ可算な開集合族とする. 任意の$ n \in \N $に対して$ V_n \defeq U_0 \cap U_1 \cap \cdots \cap U_n $と定めれば, $ \{V_n\}_{n \in \N} $は空でない開集合からなる下降列なので$ \bigcap_{n \in \N} \topcl U_n \neq \emptyset $である.

	(4) $ \Longrightarrow $ (1) : もし$ X $が擬コンパクトでなければ, 有界でない連続関数$ f \colon X \to \R $が存在する. ここで, 任意の$ n \in \N $に対して$ U_n \defeq \setcomp{x \in X}{|f(x)| > n}$と定めると$ \{U_n\}_{n \in \N} $は空でない開集合からなる下降列であって$ \bigcap_{n \in \N} \topcl U_n = \emptyset $であることから仮定に矛盾する.
\end{proof}

\begin{proposition}
	任意の可算コンパクト空間は擬コンパクトである.
\end{proposition}

\begin{proof}
	可算コンパクト空間の連続像が可算コンパクトであることと, $ \R $における可算コンパクト部分集合がコンパクトであることからわかる.
\end{proof}

\begin{lemma}
	\label{lamma:A perfect mapping preserves locally finite property}
	$ f \colon X \to Y $を完全写像とし, $ \mathscr{A} $を$ X $の任意の局所有限な集合族とする. このとき, $ Y $の集合族$ \setcomp{f[A]}{A \in \mathscr{A}} $は局所有限である.
\end{lemma}

\begin{proof}
	任意の点$ y \in Y $に対して, $ f^{-1}(y) $に属する任意の点$ x $はその開近傍$ U_x $であって$ \setcomp{A \in \mathscr{A}}{A \cap U_x \neq \emptyset} $が有限となるものをもつ. このとき, $ \setcomp{U_x}{x \in f^{-1}(y)} $は$ f^{-1}(y) $の開被覆であり, $ f^{-1}(y) $がコンパクトであることから有限部分集合$ B \subset f^{-1}(y) $であって$ V \defeq \bigcup_{x \in B} U_x $が有限個の$ \mathscr{A} $の元としか交わらず, なおかつ$ f^{-1}(y) \subset V $を満たすものが存在する. いま、$ f $は閉写像なので, $ f[X \setminus V] $は$ y $を含まない閉集合である. ゆえに, $ y $の開近傍$ W $であって$ f[X \setminus V] $と交わらないものが存在し, $ f^{-1}[W] \subset f^{-1}[Y \setminus f[X \setminus V]] = X \setminus f^{-1}f[X \setminus V] \subset V $となる. したがって, $ W $は$ \setcomp{f[A]}{A \in \mathscr{A}} $の有限個の元としか共通部分をもたない.

\end{proof}

\begin{lemma}
	\label{lemma:Lemma for the proof that every product of pseudocompact space X and pseudocompact k space is also pseudocompact in a class of T3.5 space}
	無限集合$ \Lambda $に対し, $ \{A_\lambda \times B_\lambda\}_{\lambda \in \Lambda} $を
	$ X \times Y $における非空部分集合からなる局所有限な集合族とする.
	ここで, $ X $は \Hausdorff 空間であり, $ Y $は\Hausdorff なk-空間である.
	このとき, 無限部分集合$ \Lambda_0 \subset \Lambda $であって,
	$ \{A_\lambda\}_{\lambda \in \Lambda_0} $と$ \{B_\lambda\}_{\lambda \in \Lambda_0} $の少なくとも一方が局所有限となるものが存在する.
\end{lemma}

\begin{proof}
	$ \{\topcl B_\lambda\}_{\lambda \in \Lambda} $が局所有限であれば$ \Lambda_0 $として$ \Lambda $をとればよいので
	$ \{\topcl B_\lambda\}_{\lambda \in \Lambda} $が局所有限でない場合について示す.
	$ Y $のコンパクト部分集合$ Z $と無限集合$ \Lambda_0 \subset \Lambda $であって,
	$ Z \cap \topcl B_{\lambda} \neq \emptyset $が全ての$ \lambda \in \Lambda_0 $に対して成り立つものが存在することを示せばよい.
	なぜなら, 系\ref{coro:Kuratowski-Mrowka Theorem}より射影$ p \colon X \times Z \to X $が完全写像であることから,
	補題\ref{lamma:A perfect mapping preserves locally finite property}より
	局所有限な集合族$ \{A_\lambda \times (Z \cap \topcl B_\lambda)\}_{\lambda \in \Lambda_0} $に対して
	$ \{A_\lambda\}_{\lambda \in \Lambda_0} $が$ X $の局所有限な部分集合族となるからである.
	ここからは, そのような$ Z \subset Y $と$ \Lambda_0 \subset \Lambda $の存在を示す.
	いま, $ \{\topcl B_\lambda\}_{\lambda \in \Lambda} $が局所有限でないことから, $ y \in Y $であって,
	その任意の近傍$ U $が無限個の$ \{\topcl B_\lambda\}_{\lambda \in \Lambda} $の元と共通部分を持つものが存在する.
	ここで, $ \Lambda' = \setcomp{\lambda \in \Lambda}{y \in \Lambda} $が無限集合であれば
	$ Z \defeq \{y\}, \Lambda_0 \defeq \Lambda' $と定めればよい.
	$ \Lambda' $が有限集合の場合, $ B \defeq \bigcup_{\lambda \in \Lambda \setminus \Lambda'} \topcl B_\lambda $と定めると
	$ y $の取り方から$ y \in \topcl B $であるが$ y \notin B $なので$ B $は閉集合でない.
	よって, $ Y $がk-空間であることから, あるコンパクト部分集合$ K \subset Y $であって$ K \cap B $が$ K $の閉集合とならないものが存在する.
	このとき, $ Z \defeq K, \Lambda_0 \defeq \setcomp{\lambda \in \Lambda \setminus \Lambda'}{Z \cap \topcl B_\lambda} $と定めればよい.
	なぜなら, $ \Lambda_0 $がもし仮に有限ならば
	$ Z \cap B = \bigcup_{\lambda \in \Lambda \setminus \Lambda'} Z \cap \topcl B_\lambda
	= \bigcup_{\lambda \in \Lambda_0} \topcl Z \cap B_\lambda $より
	$ Z \cap B $が$ Z $の閉集合となり矛盾するので$ \Lambda_0 $が無限集合でなければならないからである.
\end{proof}

\begin{proposition}
	\topT{3.5}空間のクラスにおいて, 擬コンパクト空間$ X $と擬コンパクトなk-空間$ Y $の積空間$ X \times Y $は擬コンパクトである.
\end{proposition}

\begin{proof}
	$ \mathscr{U} $を$ X \times Y $の非空開集合からなる局所有限な任意の開被覆とする. $ \mathscr{U} $が有限被覆でないと仮定して矛盾を導く. 任意の$ U \in \mathscr{U} $に対して, 空でない開集合$ V_U \in X, W_U \in Y $が存在して$ V_U \times W_U \subset U $が成り立つ. このとき, $ \mathscr{U}' \defeq \setcomp{V_U \times W_U}{U \in \mathscr{U}} $は局所有限な空でない開集合からなる族である. $ \mathscr{U}' $が有限であるとき, $ \{V_U\}_{U \in \mathscr{V}} $と$ \{W_U\}_{U \in \mathscr{V}} $は局所有限である. また, $ \mathscr{U}' $が有限でない場合, 補題\ref{lemma:Lemma for the proof that every product of pseudocompact space X and pseudocompact k space is also pseudocompact in a class of T3.5 space}から無限部分集合$ \mathscr{V} \subset \mathscr{U} $が存在して$ \{V_U\}_{U \in \mathscr{V}} $と$ \{W_U\}_{U \in \mathscr{V}} $の少なくとも一方は局所有限となる. $ \{V_U\}_{U \in \mathscr{V}} $が局所有限であるとすると($ \{W_U\}_{U \in \mathscr{V}} $が局所有限の場合についても以降と同様の議論で矛盾が導かれる), $ X $が擬コンパクトであることから$ \{V_U\}_{U \in \mathscr{V}} $は有限である. このとき, もしも$ \{W_U\}_{U \in \mathscr{V}} $が局所有限でなければ$ \mathscr{V} $が局所有限でないという矛盾が導かれるので$ \{W_U\}_{U \in \mathscr{V}} $は局所有限である. ゆえに, $ Y $が擬コンパクトであることから$ \{W_U\}_{U \in \mathscr{V}} $は有限である. よって, $ \mathscr{V} $が有限集合であることになり矛盾する.
\end{proof}

\begin{corollary}
	\topT{3.5}空間のクラスにおいて, 擬コンパクト空間$ X $とコンパクト空間$ Y $の積空間$ X \times Y $は擬コンパクトである.
\end{corollary}

\begin{proposition}
	任意のハイパー連結空間およびウルトラ連結空間は擬コンパクトである.
\end{proposition}

\begin{proof}
	$ f \colon X \to R $をハイパー連結空間$ X $上の実数値連続関数とする. $ X $は連結なので$ f[X] $は$ \R $の連結集合である. もし, $ f[X] $が2点以上の点を含むと仮定すると$ f[X] $は$ \R $のある開区間$ \intoo{a}{b} $($ a, b \in \R, a < b$)を含む. よって, $ \intoo{a}{b} $に含まれる互いに交わらない開集合$ U, V $を取ることができる. このとき, $ f^{-1}[U] \cap f^{-1}[V] = \emptyset $となりハイパー連結であることに矛盾する. ゆえに, $ f $は定値写像である. ウルトラ連結の場合についても同様の証明である.
\end{proof}

\end{document}
