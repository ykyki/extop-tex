\documentclass[uplatex, dvipdfmx, a4paper, 12pt, class=jsbook, crop=false]{standalone}
\usepackage{import}
\import{../}{common-preamble.sty}

\begin{document}
\section{基数関数}
\label{sec:cardinal-functions}

\begin{definition}
	位相空間$ X $の \indexe{density} とは, 基数
	\[ \topdensity{X} \defeq \min \setcomp{\cardinality{D}}{\text{$ D $は$ X $の稠密部分集合である}}\]
	のことである.
\end{definition}

\begin{definition}
	$ X $の(位相的)\indexjj{うぇいと}{ウェイト}{weight}とは, 基数
	\[ \topweight{X} \defeq \min \setcomp{\cardinality{\topbasis}}{\text{$ \topbasis $は$ X $の開基である}} \]
	のことである.
\end{definition}

\begin{definition}
	$ X $の\indexjj{ねっとわーくのうど}{ネットワーク濃度}{network weight}とは, 基数
	\[ \topnetworkweight{X} \defeq \min \setcomp{\cardinality{\mathscr{N}}}{\text{$ \mathcal{N} $は$ X $のネットワークである}} \]	
	のことである.
	ただし, $ X $の部分集合族$ \mathscr{N} $が\indexjj{ねっとわーく}{ネットワーク}{network}であるとは, 任意の点$ x \in X $とその任意の近傍$ U $に対して$ x \in N \subset U $なる$ N \in \mathscr{N} $が存在することをいう.
\end{definition}

\begin{definition}
	位相空間$ X $の点$ x $の\indexjj{きゃらくたー}{キャラクター}{character}とは, 基数
	\[ \topcharacter{x} \defeq \min \setcomp{\cardinality{\mathcal{N}}}{\text{$ \mathcal{N} $は$ x $の近傍基である}} \]
	のことである.
	また, 位相空間$ X $のキャラクターとは, 基数
	\[ \topcharacter{X} \defeq \sup \setcomp{\topcharacter{x}}{x \in X} \]
	のことである.
\end{definition}

\begin{proposition}
	位相空間$ X $と基数$\kappa$について$ \topdensity{X} \leq \kappa $が成り立ち, かつ連続な全射$ f \colon X \rightarrow Y $が存在するとする.
	このとき$ \topdensity{Y} \leq \kappa $が成り立つ.
	\qed
\end{proposition}

\end{document}
