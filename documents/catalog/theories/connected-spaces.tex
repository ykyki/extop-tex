\documentclass[uplatex, dvipdfmx, a4paper, 12pt, class=jsbook, crop=false]{standalone}
\usepackage{import}
\import{../}{common-preamble.sty}

\begin{document}
\section{連結性}
\label{sec:connected-spaces}

\newcommand{\longrel}[1]{\ #1\ }

\begin{source}
	本節は主に\cite[Chapter 6]{Engelking1989GT}や\cite[第4章]{Morita1981ja}を参考にしている.
\end{source}

\begin{definition}
	位相空間$ X $が\indexjj{れんけつ}{連結}{connected}であるとは, $ X $を有限個の互いに交わらない開集合の合併で表したとき常に,
	それら開集合の中で少なくとも1つが$ X $と等しくなることである.
\end{definition}

空集合は0個の空集合の合併であるから, 定義より空集合は連結ではない.

\begin{proposition}
	\label{prop:Characterization of connectedness}
	位相空間$ X $について以下の条件は同値である.
		\begin{enumerate}
			\item $ X $が連結である.
			\item $ X $が空でなく, さらに次が成り立つ: $ X $が互いに交わらない2つの開集合$ G_0, G_1 $の合併として表せるならば,
			      $ G_0 $か$ G_1 $のどちらかが$ X $に等しい.
			\item $ X $が空でなく, さらに次が成り立つ: $ X $が互いに交わらない2つの閉集合$ F_0, F_1 $の合併として表せるならば,
			$ F_0 $か$ F_1 $のどちらかが$ X $に等しい.
			\item $ X $が空でなく, かつ$ X $の開閉集合が$ \emptyset $と$ X $のみである.
			\item $ X $が空でなく, かつ$ X $の部分集合であってその境界が空集合になるものが$ \emptyset $と$ X $のみである.
			\item $ X $から離散空間$ D $への連続全射が存在するならば, $ D $は一点集合である.
		\end{enumerate}
\end{proposition}

\begin{proof}
	定義から容易に確かめられる.
\end{proof}

\begin{proposition}
	\label{prop:Continuous maps preserve connectedness}
	連結性は連続写像によって保たれる.
\end{proposition}

\begin{proof}
	$ \morph{f}{X}{Y} $を連結空間$ X $から位相空間$ Y $への連続な全射とする.
	$ \morph{g}{Y}{D} $を離散空間$ D $への連続全射とすると$ \morph{g \compo f}{X}{D} $も
	連続全射であり, $ D $は一点集合である.
\end{proof}

\begin{proposition}
	\label{prop:PathCtd>Ctd}
	位相空間$ X $が弧状連結ならば連結である.
\end{proposition}

\begin{proof}
	弧状連結性から$ X $は空でない.
	位相空間$ X $が互いに交わらない2つの閉集合$ F_0, F_1 $の合併で表されているとする.
	ここで, $ F_0, F_1 $がともに空集合でないと仮定して矛盾を導く.
	2点$ x_0 \in F_0, x_1 \in F_1 $をとると, $ X $の弧状連結性より連続写像$ \morph{\gamma}{\I}{X} $であって,
	$ \mappt{\gamma}{0} = x_0, \mappt{\gamma}{1} = x_1 $を満たすものが存在する.
	このとき, $ \I $内の点列$ (a_n)_{n\in \N}, (b_n)_{n \in \N} $を帰納的に次のように定める.
	\begin{itemize}
		\item $ a_0 = 0, b_0 = 1 $.
		\item $ n \geq 1 $について,
		\begin{equation}
			\begin{cases}
				a_n = \frac{a_{n-1} + b_{n-1}}{2}, b_n = b_{n-1} & \gamma\left(\frac{a_{n-1} + b_{n-1}}{2}\right) \in F_0 \text{のとき} \\
				a_n = a_{n-1}, b_n = \frac{a_{n-1} + b_{n-1}}{2} & \gamma\left(\frac{a_{n-1} + b_{n-1}}{2}\right) \in F_1 \text{のとき} .
			\end{cases}
		\end{equation}
	\end{itemize}
	定義から$ a_0 \leq a_1 \leq \cdots $と$ b_0 \geq b_1 \geq \cdots $が成り立つ.
	また, 任意の$ n \in \N $について$ b_n - a_n = 2^{-n} $が成り立つ.
	よって, 完備性よりある点$ c \in \I $が存在して$ \lim a_n = c, \lim b_n = c $が成り立つ.
	$ \setcomp{a_n}{n \in \N} \subset \mapinvset{\gamma}{F_0}, \setcomp{b_n}{n \in \N} \subset \mapinvset{\gamma}{F_1}$であることから,
	$ c \in \topcl{\mapinvset{\gamma}{F_0}} $かつ$ c \in \topcl{\mapinvset{\gamma}{F_1}} $である.
	$ F_0, F_1 $は閉集合なので$\topcl{\mapinvset{\gamma}{F_0}} = \mapinvset{\gamma}{F_0}, \topcl{\mapinvset{\gamma}{F_1}} = \mapinvset{\gamma}{F_1} $であり,
	$ \mapinvset{\gamma}{F_0} \cap \mapinvset{\gamma}{F_1} = \emptyset $に矛盾する.
\end{proof}

\begin{proposition}
	\label{c00001}
	位相空間$ X $の連結部分集合からなる族$ \setfamily{A_\lambda}{\lambda \in \Lambda} $が
	$ \bigcap_{\lambda \in \Lambda} A_\lambda \neq \emptyset $を満たすならば
	$ \bigcup_{\lambda \in \Lambda} A_\lambda $は連結である.
\end{proposition}

\begin{proof}
	$ D $を離散空間とし, $ \morph{f}{\bigcup_{\lambda \in \Lambda} A_\lambda}{D} $を
	連続な全射とする. ここで, $ x_0 \in \bigcap_{\lambda \in \Lambda} A_\lambda $
	と定めると, 任意の$ \lambda \in \Lambda $に対して$ A_\lambda $が連結であることから
	\cref{prop:Characterization of connectedness}より任意の$ x \in A_\lambda $に対して
	$ \mappt{f}{x} = \mappt{f}{x_0} $となる.
	よって, $ \mapset{f}{\bigcup_{\lambda \in \Lambda} A_\lambda}= D $は一点集合$ \{\mappt{f}{x_0}\} $である.
\end{proof}

\begin{corollary}
	\label{c00002}
	$ X $を空でない位相空間とする.
	$ X $の任意の2点についてそれらを含む連結部分集合が存在するとき$ X $は連結である.
\end{corollary}

\begin{proof}
	$ X $の点$ x_0 $を1つとり固定する. 任意の点$ x \in X $に対して$ \{x_0, x\} \subset A_x $を満たす
	$ X $の連結部分集合が存在する. このとき, $ \setfamily{A_x}{x \in X} $は
	$ \bigcap_{x \in X} A_x \neq \emptyset$を満たす. よって, $ X = \bigcup_{x \in X} A_x $は連結である.
\end{proof}

\begin{proposition}
	\label{c00003}
	位相空間$ X $の部分集合$ A $が連結ならば, $ A \subset B \subset \topcl A $を満たす部分集合$ B $も連結である.
\end{proposition}

\begin{proof}
	$ B \neq A $の場合を示す. $ D $を離散空間とし, $ \morph{f}{B}{D} $を連続な全射とする.
	$ D $が一点集合でないと仮定して矛盾を導く.
	$ \mapset{f}{A} $は一点集合なので, $ x \in B \setminus A $であって
	$ \mappt{f}{b} \notin \mapset{f}{A} $を満たすものが存在する.
	このとき, $ \mapinvpt{f}{\mappt{f}{b}} $は$ \mapinvpt{f}{\mappt{f}{b}} \cap A = \emptyset $なる$ B $の開集合である.
	これは$ \topcl_B A = B $に矛盾する.
\end{proof}

\begin{proposition}[{\cite[定理~14.11]{Morita1981ja}}]
	\label{p00011}
	$ \setfamily{X_\lambda}{\lambda \in \Lambda} $を位相空間の族,
	$ X \defeq \prod_{\lambda \in \Lambda} X_\lambda $を積空間,
	$ \morph{p_\lambda}{X}{X_\lambda} $を射影とする.
	ここで$ X $の1点$ p = (p_\lambda)_{\lambda \in \Lambda} $をとり,
	\[ A_p \defeq \setcomp{(x_\lambda)_{\lambda \in \Lambda} \in X}
	{\setcomp{\lambda \in \Lambda}{x_\lambda \neq p_\lambda} \text{が有限}} \]
	と定める. このとき, $ X = \topcl A_p $が成り立つ.
\end{proposition}

\begin{proof}
	$ X $の任意の点$ x = (x_\lambda)_{\lambda \in \Lambda} $をとり,
	$ x $の任意の開近傍が$ A_p $と交わることを示せば良い.
	積位相の定義より,
	\[ \setcomp{\bigcap_{\lambda \in \Lambda'} \mapinvset{p_\lambda}{U_\lambda}}
	{\text{$\Lambda'$は$\Lambda$の有限部分集合},
	\text{$U_\lambda$は$X_\lambda$の開近傍}} \]
	は$ x $の近傍基をなすことがわかる.
	有限集合$ \Lambda' \subset \Lambda $に対して$ y = (y_\lambda)_{\lambda \in \Lambda} $を
	次のように定義する:
	\[ y_\lambda \defeq \begin{cases}
		x_\lambda & \lambda \in \Lambda' \text{のとき} \\
		p_\lambda & \lambda \notin \Lambda' \text{のとき}.
	\end{cases} \]
	このとき, $ y \in A_p \cap \lrbrack{\bigcap_{\lambda \in \Lambda'}
	\mapinvset{p_\lambda}{U_\lambda}} $より,
	$ x $の任意の開近傍が$ A_p $と交わることが示された.
\end{proof}

\begin{proposition}
	\label{c00004}
	$ \setfamily{X_\lambda}{\lambda \in \Lambda} $を位相空間の族とする.
	積空間$ X \defeq \prod_{\lambda \in \Lambda} X_\lambda $が連結であることと,
	任意の$ \lambda \in \Lambda $に対して$ X_\lambda $が連結であることは同値である.
\end{proposition}

\begin{proof}
	$ X $が連結ならば各$ X_\lambda $が連結であることは,
	射影$ \morph{p_\lambda}{X}{X_\lambda} $が連続全射であることからわかる.
	以降では逆を示す.
	まず, $ \Lambda $が有限の場合について証明する.
	一般性を失うことなく$ \Lambda = \{1, 2, \ldots, n\} $と仮定してよい.
	$ n $に関する帰納法で示す.
	$ n = 0 $, すなわち$ \Lambda = \emptyset $の場合,
	一点からなる空間は連結なので, $ X $は連結である.
	$ n-1 $のとき主張が成り立つと仮定する.
	任意に$ X $の2点$ x = (x_1, \ldots, x_n), y = (y_1, \ldots, y_n) $をとる.
	帰納法の仮定より$ A \defeq \lrbrace{x_1} \times X_2 \times \cdots \times X_n,
	B \defeq X_1 \times \cdots \times X_{n-1} \times \lrbrace{y_n} $は
	連結部分集合である. $ (x_1, x_2, \ldots, x_{n-1}, y_n) \in A \cap B $なので,
	\cref{c00001}より$ A \cup B $は連結集合である.
	これは2点$ x, y $を含むので\cref{c00002}より
	$ X = X_1 \times \cdots \times X_n $は連結である.

	次に, $ \Lambda $が一般の場合について示す.
	$ X $の1点$ p $をとる.
	$ A_p $を\cref{p00011}と同様に定義する.
	$ A_p $が連結であることを示せば,
	\cref{c00003}より$ X $は連結である.
	$ \mathscr{F} $を$ \Lambda $の有限部分集合全体からなる集合族とする.
	$ \Lambda' \in \mathscr{F} $に対して
	$ A_{\Lambda'} \defeq \setcomp{(x_\lambda)_{\lambda \in \Lambda} \in X }
	{\text{任意の$\lambda \in \Lambda \setminus \Lambda' $について$ x_\lambda = p_\lambda$}} $と定めると
	$ A_p = \bigcup_{\Lambda' \in \mathscr{F}} A_{\Lambda'} $と表せる.
	$ \Lambda $が有限の場合の証明より,
	各$ \Lambda' \in \mathscr{F} $について$ A_{\Lambda'} $は連結である.
	また, 任意の$ \Lambda' \in \mathscr{F} $に対して$ p \in  A_{\Lambda'} $なので, \cref{c00001}より
	$ A_p $は連結である.
\end{proof}

\begin{definition}
	位相空間$ X $の点$ x $に対して,
	\[ C(x) \defeq \bigcup \setcomp{C \in \pow(X)}{x \in C, C\mbox{は連結集合}} \]
	で定義される$ C(x) $を$ X $における点$ x $の
	\indexjj{れんけつせいぶん}{連結成分}{connected component}という.
\end{definition}

\begin{proposition}
	\label{c00005}
	位相空間$ X $の点$ x $に対して, 連結成分$ C(x) $は$ x $を含む最大の連結集合であり,
	閉集合である.
\end{proposition}

\begin{proof}
	\cref{c00001}より$ C(x) $は$ x $を含む最大の連結集合である.
	また, \cref{c00003}より$ \topcl C(x) $も連結集合である.
	$ C(x) $の最大性より$ \topcl C(x) \subset C(x) $が成り立つので, $ C(x) $は閉集合である.
\end{proof}

\begin{proposition}
	\label{c00006}
	位相空間$ X $における2点$ x, y $の連結成分$ C(x), C(y) $について,
	$ C(x) = C(y) $と$ C(x) \cap C(y) = \emptyset $のいずれか一方が成り立つ.
\end{proposition}

\begin{proof}
	$ C(x) \cap C(y) \neq \emptyset $のとき,
	\cref{c00001}より$ C(x) \cup C(y) $は連結集合である.
	$ C(x) $は$ x $を含む最大の連結集合なので,
	$ C(x) \cup C(y) \subset C(x) $より$ C(y) \subset C(x) $が成り立つ.
	また, $ C(y) $についてもその最大性から$ C(x) \subset C(y) $が成り立つので
	$ C(x) = C(y) $である.
\end{proof}

\begin{proposition}[{\cite[Theorem~6.1.21]{Engelking1989GT}}]
	\label{c00007}
	$ \setfamily{X_\lambda}{\lambda \in \Lambda} $を位相空間の族とする.
	積空間$ X \defeq \prod_{\lambda \in \Lambda} X_\lambda $の
	点$ x \defeq (x_\lambda)_{\lambda \in \Lambda} $の連結成分を$ C(x) $とする.
	また, 各$ \lambda \in \Lambda $について
	$ X_\lambda $における$ x_\lambda $の連結成分を$ C_\lambda $と書くことにする.
	このとき, $ C(x) = \prod_{\lambda \in \Lambda} C_\lambda $が成り立つ.
\end{proposition}

\begin{proof}
	各$ \lambda \in \Lambda $について$ \morph{p_\lambda}{X}{X_\lambda} $を射影とする.
	$ \mapset{p_\lambda}{C} $は$ x_\lambda $を含む連結集合なので
	$ \mapset{p_\lambda}{C} \subset C_\lambda $が成り立つ.
	よって, $ C \subset \prod_{\lambda \in \Lambda} C_\lambda $が成り立つ.
	また, \cref{c00004}より$ \prod_{\lambda \in \Lambda} C_\lambda $は連結集合なので
	$ C $の最大性より$ C = \prod_{\lambda \in \Lambda} C_\lambda $となる.
\end{proof}

\begin{definition}
	位相空間$ X $における点$ x $に対して,
	\[ Q(x) \defeq \bigcap \setcomp{Q \in \pow(X)}{x \in Q, Q\mbox{は開閉集合}} \]
	で定義される$ Q(x) $を$ X $における点$ x $の\indexjj{ぎれんけつせいぶん}
	{擬連結成分}{quasi-component}という.
\end{definition}

\begin{proposition}[{\cite[Theorem~6.1.22]{Engelking1989GT}}]
	\label{c00008}
	位相空間$ X $における点$ x $の連結成分$ C(x) $と擬連結成分$ Q(x) $について
	$ C(x) \subset Q(x) $が成り立つ.
\end{proposition}

\begin{proof}
	$ F \subset X $を$ x $を含む開閉集合とする.
	部分空間$ C(x) $において$ C(x) \cap F $と$ C(x) \setminus F $は
	それぞれ交わらない閉集合である.
	$ C(x) $が連結なので$ C(x) \cap F $と$ C(x) \setminus F $のいずれか一方は
	空集合である. 今, $ x \in C(x) \cap F $より$ C(x) \setminus F = \emptyset $である.
	よって, $ C(x) \subset F $より$ C(x) \subset Q(x) $である.
\end{proof}

\begin{lemma}
	\label{c00009}
	$ U $を位相空間$ X $の開集合, $ \setfamily{F_\lambda}{\lambda \in \Lambda} $を
	閉集合の族($ \Lambda \neq \emptyset $)とする.
	$ F_\lambda $の中で少なくともひとつはコンパクトであるとする
	(とくに$ X $がコンパクトであればこの条件は満たされる).
	また$ \bigcap_{\lambda} F_\lambda \subset U $となっているとする.
	このとき有限個の$ F_1, \ldots , F_n \in \{ F_\lambda\}$が存在して$ F_1 \cap \cdots \cap F_n \subset U $となる.
\end{lemma}

\begin{proof}
	コンパクトである$ F_0 \in \{F_\lambda\}$を取る.
	すると$ U \cap F_0 $は$ F_0 $の開集合,
	$ \setfamily{F_0 \cap F_\lambda}{\lambda \in \Lambda} $は$ F_0 $の閉集合の族であり,
	$ \bigcap_{\lambda} (F_0 \cap F_\lambda) \subset U \cap F_0 $となる.
	そこで最初から$ X $がコンパクトである条件のもとで証明すれば十分である.

	$ X $がコンパクトであるとする.
	いまコンパクト部分空間$ X \setminus U $が開集合の族$ (X \setminus F_\lambda) $で覆われているので,
	有限個の$ F_1, \ldots, F_n $が存在して
	\[ X \setminus U \subset \bigcup_{i=1}^{n} X \setminus F_i \]
	となる. よって$ F_1 \cap \cdots \cap F_n \subset U $である.
\end{proof}

\begin{lemma}[{\cite[Theorem~6.1.23]{Engelking1989GT}}]
	\label{c00010}
	$ X $をコンパクト\Hausdorff 空間, $ x_0 $を$ X $の点とする.
	$ x_0 $の連結成分を$ C $, 擬連結成分を$ Q $とおくと, $ C = Q $である.
\end{lemma}

\begin{proof}
	$ x_0 \in K $となる開閉集合$ K $全体の集合を$ \mathscr{K} $とおく. いま$ Q  = \bigcap \mathscr{K}$である.

	($ \subset $): $ K \in \mathscr{K} $を任意に取る.
	$ C \cap K $と$ C \setminus K $がともに$ C $の開閉集合であり,
	$ C \cap K \neq \emptyset $となるので, $ C $の連結性より$ C \subset K $である.
	よって$ C \subset Q $である.

	($ \supset $): $ Q $が連結であることを示せばよい.
	$ Q_1, Q_2 $を$ Q_1 \cap Q_2 = \emptyset, Q_1 \cup Q_2 = Q , x_0 \in Q_1$となる$ Q $の任意の閉集合とする.
	$ Q $が閉集合であるから, $ Q_1, Q_2 $は$ X $においても閉である.
	$ X $が \topT{4} 空間であるから, 開集合$ U_1, U_2 \subset X $が存在して
	$ Q_1 \subset U_1, Q_2 \subset U_2, U_1 \cap U_2 = \emptyset $となる.
	$ \bigcap \mathscr{K} = Q \subset U_1 \cup U_2 $に\cref{c00009}を適用して,
	有限個の$ K_1 , \ldots, K_n \in \mathscr{K}$で$ \bigcap_{1}^{n} K_i \subset U_1 \cup U_2$となるものが存在する.
	ここで$ K_0 \defeq \bigcap_{1}^{n} K_i $とおけば$ K_0 \in \mathscr{K}, Q \subset K_0 \subset U_1 \cup U_2 $となる.
	すると
	\[ \topcl \lrparen{U_1 \cap K_0} \longrel{\subset} \lrparen{\topcl U_1} \cap K_0
	\longrel{=} \left(\topcl U_1 \right) \cap (U_1 \cup U_2) \cap K_0 \longrel{=} U_1 \cap K_0\]
	となるので, $ U_1 \cap K_0 \in \mathscr{K} $である.よって
	\[ Q_2 \longrel{\subset} Q \longrel{\subset} U_1 \cap K_0 \longrel{\subset} U_1 \]
	となり, $ Q_2 \subset U_1 \cap U_2 $が成り立つ.
	よって$ U_1 \cap U_2 = \emptyset, U_1 \supset Q_1 \neq \emptyset $より
	$ Q_2 = \emptyset $である. ゆえに$ Q $は連結である.
\end{proof}

\begin{lemma}[{\cite[Theorem~6.1.25]{Engelking1989GT}}]
	\label{c00011}
	$ A $を連続体$ X $の閉集合で$ A \neq \emptyset, X $となるものとする.
	空間$ A $の任意の連結成分$ C $に対し, $ C \cap \topbry A \neq \emptyset $が成立する.
\end{lemma}

\begin{proof}
	点$ x_0 \in C $を固定する. $ x_0 \in K $となる$ A $の開閉集合$ K $全体の集合を$ \mathscr{K} $とおく.
	補題\cref{c00010}より$ C = \bigcap \mathscr{K} $である. もし仮に$ C \cap \topbry A = \emptyset $であったとする.

	$ \topbry A \subset A $は$ A $の閉集合であるからコンパクトである. 仮定より
	\[ \bigcap_{K \in \mathscr{K}} K \cap \topbry A = C \cap \topbry A = \emptyset \]
	なので, コンパクト性と併せて$ K \cap \topbry A = \emptyset $となる$ K \in \mathscr{K} $の存在が分かる.
	$ K = U \cap A  $となる$ X $の開集合$ U $を取る. $ A = \topint A \cup \topbry A $と分割されているので,
	$ K = U \cap \topint A $である. よって$ K $は$ X $においても開閉集合である.
	$ x_0 \in K $と$ X $の連結性より$ K = X $である.
	よって$ K \subset A $より$ A = X $となり, $ A $の仮定に矛盾する.
\end{proof}

\begin{lemma}{\cite[Lemma~6.1.26]{Engelking1989GT}}]
	\label{c00012}
	連続体$ X $の互いに交わらない閉被覆$ \setfamily{F_i}{i \in \N} $が与えられており,
	$ F_i $の中でも少なくとも2つは非空であるとする.
	このときある連続体$ C \subset X $が存在して$ C \cap F_0 = \emptyset $となり,
	さらに$ C \cap F_1, C \cap F_2, \ldots $の中でも少なくとも2つは非空となる.
\end{lemma}

\begin{proof}
	$ F_n \neq \emptyset $となる$ n \neq 0 $を取る. $ X $が正規空間なので,
	開集合$ U, V \subset X $で$ F_0 \subset U, F_n \subset V, U \cap V = \emptyset $となるものが存在する.
	点$ x \in F_n $と連結成分$ x \in C \subset F_n $を取る.
	この$ C $が所望の条件を満たす連続体になる: まず$ C \cap F_0 = \emptyset, C \cap F_n \neq \emptyset $が成立する.
	\cref{c00011}より, $ C \cap \topbry\topcl V \neq \emptyset $なので, この集合の元$ y $が存在する.
	また$ V $の取り方から$ F_n \subset \topint \topcl V $である.
	よって, $ y \notin F_n $である.
	そこで$ y \in F_k $となる$ k $を取れば$ k \neq n $かつ$ C \cap F_k \neq \emptyset $となる.
\end{proof}

\begin{definition}
	位相空間$ X $が\indexjj{しぐまれんけつ}{\sigmaConnected}{\ensuremath{\sigma}-connected}であるとは,
	$ X $を可算個の互いに交わらない閉集合の合併で表したとき常に,
	それら閉集合の中で少なくとも1つが$ X $と等しくなることである.
\end{definition}

\begin{theorem}[\Sierpinski の定理]
	\label{c00013}
	連続体$X$は\sigmaConnected である.
\end{theorem}

\begin{proof}
	$ X $の互いに交わらない閉被覆$ \setfamily{F_i}{i \in \N} $が与えられているとする.
	もし仮に$ F_i \neq \emptyset $となる$ i $が少なくとも2つ存在したとする.
	\cref{c00012}よりある連続体$ C_0 \subset X $が存在して$ C_0 \cap  F_0 = \emptyset$となり,
	さらに$ C_0 \cap F_1, C_0 \cap F_2, \ldots $の中でも少なくとも2つは非空となる.
	\cref{c00012}を再び$ C_0 $と閉被覆$ \setfamily{C_0 \cap F_i}{i \in \N_{>0}} $について適用する.
	するとある連続体$ C_1 \subset C_0 $が存在して$ C_1 \cap  F_1 = \emptyset$となり,
	さらに$ C_1 \cap F_2, C_1 \cap F_3, \ldots $の中でも少なくとも2つは非空となる.
	この操作を繰り返すことで, 連続体の下降列$ C_0 \supset C_1 \supset \cdots $で
	$ C_i \cap F_i = \emptyset, C_i \neq \emptyset $が任意の$ i \in \N $について成立するものが得られる.
	$ X $がコンパクトなので$ \bigcap_{i=0}^{\infty} C_i \neq \emptyset $である.
	しかし一方で$ \bigcap_{i=0}^{\infty} C_i
	= \left(\bigcap_{i=0}^{\infty} C_i \right) \cap \left( \bigcup_{i=0}^{\infty} F_i \right) = \emptyset $ である. よって矛盾する.
\end{proof}

\end{document}
