\documentclass[uplatex, dvipdfmx, a4paper, 12pt, class=jsbook, crop=false]{standalone}
\usepackage{import}
\import{../}{common-preamble.sty}

\begin{document}
\section{連結性}
\label{sec:connected-spaces}

\begin{source}
	本節は主に\cite[Chapter 6]{Engelking1989GT}や\cite[第4章]{Morita1981ja}を参考にしている.
\end{source}

\begin{definition}
	位相空間$ X $が\indexjj{れんけつ}{連結}{connected}であるとは, $ X $を有限個の互いに交わらない開集合の合併で表したとき常に,
	それら開集合の中で少なくとも1つが$ X $と等しくなることである.
\end{definition}

空集合は0個の空集合の合併であるから, 定義より空集合は連結ではない.

\begin{proposition}
	\label{prop:Characterization of connectedness}
	位相空間$ X $について以下の条件は同値である.
		\begin{enumerate}
			\item $ X $が連結である.
			\item $ X $が空でなく, さらに次が成り立つ: $ X $が互いに交わらない2つの開集合$ G_0, G_1 $の合併として表せるならば,
			      $ G_0 $か$ G_1 $のどちらかが$ X $に等しい.
			\item $ X $が空でなく, さらに次が成り立つ: $ X $が互いに交わらない2つの閉集合$ F_0, F_1 $の合併として表せるならば,
			$ F_0 $か$ F_1 $のどちらかが$ X $に等しい.
			\item $ X $が空でなく, かつ$ X $の開閉集合が$ \emptyset $と$ X $のみである.
			\item $ X $が空でなく, かつ$ X $の部分集合であってその境界が空集合になるものが$ \emptyset $と$ X $のみである.
			\item $ X $から離散空間$ D $への連続全射が存在するならば, $ D $は一点集合である.
		\end{enumerate}
\end{proposition}

\begin{proof}
	定義から容易に確かめられる.
\end{proof}

\begin{proposition}
	\label{prop:Continuous maps preserve connectedness}
	連結性は連続写像によって保たれる.
\end{proposition}

\begin{proof}
	$ \morph{f}{X}{Y} $を連結空間$ X $から位相空間$ Y $への連続な全射とする.
	$ \morph{g}{Y}{D} $を離散空間$ D $への連続全射とすると$ \morph{g \compo f}{X}{D} $も
	連続全射であり, $ D $は一点集合である.
\end{proof}

\begin{proposition}
	\label{prop:PathCtd>Ctd}
	位相空間$ X $が弧状連結ならば連結である.
\end{proposition}

\begin{proof}
	弧状連結性から$ X $は空でない.
	位相空間$ X $が互いに交わらない2つの閉集合$ F_0, F_1 $の合併で表されているとする.
	ここで, $ F_0, F_1 $がともに空集合でないと仮定して矛盾を導く.
	2点$ x_0 \in F_0, x_1 \in F_1 $をとると, $ X $の弧状連結性より連続写像$ \morph{\gamma}{\I}{X} $であって,
	$ \mappt{\gamma}{0} = x_0, \mappt{\gamma}{1} = x_1 $を満たすものが存在する.
	このとき, $ \I $内の点列$ (a_n)_{n\in \N}, (b_n)_{n \in \N} $を帰納的に次のように定める.
	\begin{itemize}
		\item $ a_0 = 0, b_0 = 1 $.
		\item $ n \geq 1 $について,
		\begin{equation}
			\begin{cases}
				a_n = \frac{a_{n-1} + b_{n-1}}{2}, b_n = b_{n-1} & \gamma\left(\frac{a_{n-1} + b_{n-1}}{2}\right) \in F_0 \text{のとき} \\
				a_n = a_{n-1}, b_n = \frac{a_{n-1} + b_{n-1}}{2} & \gamma\left(\frac{a_{n-1} + b_{n-1}}{2}\right) \in F_1 \text{のとき} .
			\end{cases}
		\end{equation}
	\end{itemize}
	定義から$ a_0 \leq a_1 \leq \cdots $と$ b_0 \geq b_1 \geq \cdots $が成り立つ.
	また, 任意の$ n \in \N $について$ b_n - a_n = 2^{-n} $が成り立つ.
	よって, 完備性よりある点$ c \in \I $が存在して$ \lim a_n = c, \lim b_n = c $が成り立つ.
	$ \setcomp{a_n}{n \in \N} \subset \mapinvset{\gamma}{F_0}, \setcomp{b_n}{n \in \N} \subset \mapinvset{\gamma}{F_1}$であることから,
	$ c \in \topcl{\mapinvset{\gamma}{F_0}} $かつ$ c \in \topcl{\mapinvset{\gamma}{F_1}} $である.
	$ F_0, F_1 $は閉集合なので$\topcl{\mapinvset{\gamma}{F_0}} = \mapinvset{\gamma}{F_0}, \topcl{\mapinvset{\gamma}{F_1}} = \mapinvset{\gamma}{F_1} $であり,
	$ \mapinvset{\gamma}{F_0} \cap \mapinvset{\gamma}{F_1} = \emptyset $に矛盾する.
\end{proof}

\begin{proposition}
	\label{prop:Sum of every connected subsets in a family having intersection is connected}
	位相空間$ X $の連結部分集合からなる族$ \setfamily{A_\lambda}{\lambda \in \Lambda} $が
	$ \bigcap_{\lambda \in \Lambda} A_\lambda \neq \emptyset $を満たすならば
	$ \bigcup_{\lambda \in \Lambda} A_\lambda $は連結である.
\end{proposition}

\begin{proof}
	$ D $を離散空間とし, $ \morph{f}{\bigcup_{\lambda \in \Lambda} A_\lambda}{D} $を
	連続な全射とする. ここで, $ x_0 \in \bigcap_{\lambda \in \Lambda} A_\lambda $
	と定めると, 任意の$ \lambda \in \Lambda $に対して$ A_\lambda $が連結であることから
	\cref{prop:Characterization of connectedness}より任意の$ x \in A_\lambda $に対して
	$ \mappt{f}{x} = \mappt{f}{x_0} $となる.
	よって, $ \mapset{f}{\bigcup_{\lambda \in \Lambda} A_\lambda}= D $は一点集合$ \{\mappt{f}{x_0}\} $である.
\end{proof}

\begin{corollary}
	$ X $を空でない位相空間とする.
	$ X $の任意の2点についてそれらを含む連結部分集合が存在するとき$ X $は連結である.
\end{corollary}

\begin{proof}
	$ X $の点$ x_0 $を1つとり固定する. 任意の点$ x \in X $に対して$ \{x_0, x\} \subset A_x $を満たす
	$ X $の連結部分集合が存在する. このとき, $ \setfamily{A_x}{x \in X} $は
	$ \bigcap_{x \in X} A_x \neq \emptyset$を満たす. よって, $ X = \bigcup_{x \in X} A_x $は連結である.
\end{proof}

\begin{proposition}
	\label{prop:Closure of a connected subspace is connected}
	位相空間$ X $の部分集合$ A $が連結ならば, $ A \subset B \subset \topcl A $を満たす部分集合$ B $も連結である.
\end{proposition}

\begin{proof}
	$ B \neq A $の場合を示す. $ D $を離散空間とし, $ \morph{f}{B}{D} $を連続な全射とする.
	$ D $が一点集合でないと仮定して矛盾を導く.
	$ \mapset{f}{A} $は一点集合なので, $ x \in B \setminus A $であって
	$ \mappt{f}{b} \notin \mapset{f}{A} $を満たすものが存在する.
	このとき, $ \mapinvpt{f}{\mappt{f}{b}} $は$ \mapinvpt{f}{\mappt{f}{b}} \cap A = \emptyset $なる$ B $の開集合である.
	これは$ \topcl_B A = B $に矛盾する.
\end{proof}

\begin{proposition}
	\label{prop:A product space of connected spaces is connected}
	位相空間の族$ \setfamily{X_\lambda}{\lambda \in \Lambda} $について,
	その積空間$ X \defeq \prod_{\lambda \in \Lambda} X_\lambda $が連結であることと,
	任意の$ \lambda \in \Lambda $に対して$ X_\lambda $が連結であることは同値である.
\end{proposition}

\begin{proof}
	$ X $が連結ならば各$ X_\lambda $が連結であることは
	射影$ \morph{p_\lambda}{X}{X_\lambda} $が連続であることからわかる.
	以降では逆を示す.
	\WIP.
\end{proof}

\end{document}
