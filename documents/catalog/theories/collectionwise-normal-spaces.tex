\documentclass[uplatex, dvipdfmx, a4paper, 12pt, class=jsbook, crop=false]{standalone}
\usepackage{import}
\import{../}{common-preamble.sty}

\begin{document}
\section{族正規空間}
\label{sec:collectionwise-normal-spaces}

\begin{definition}
    \(X\)を位相空間とする.
    \(X\)上の集合系\(\setfamily{A_i}{i \in I}\)が\indexjj{そ}{疎}{discrete}であるとは,
    任意の点\(a \in X\)について近傍\(U\)が存在し,
    \(A_i \cap U \neq \emptyset\)となる添字\(i \in I\)が高々ひとつしか存在しないことである.
\end{definition}

\begin{definition}
    位相空間\(X\)が\indexjj{ぞくせいき}{族正規}{collectionwise normal}であるとは,
    \(X\)上の任意の疎な集合系\(\setfamily{F_i}{i \in I}\)に対し,
    開集合からなる集合系\(\setfamily{G_i}{i \in I}\)が存在して
    以下の条件をみたすことである:
    \begin{enumerate}
        \item \(\setfamily{G_i}{i \in I}\)は互いに交わらない.
        \item 任意の添字\(i \in I\)について\(F_i \subset G_i\)となる.
    \end{enumerate}
\end{definition}

\begin{proposition}
    \(X\)を位相空間,
    \(\setfamily{A_i}{i \in I}\)を集合\(X\)上の集合系とする.
    このとき以下の条件は同値である:
    \begin{enumerate}
        \item \(\setfamily{A_i}{i \in I}\)が疎である.
        \item \(\setfamily{\topcl{A_i}}{i \in I}\)が互いに交わらず, かつ局所有限である.
    \end{enumerate}
    \qed
\end{proposition}

\end{document}
