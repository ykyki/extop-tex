\documentclass[uplatex, dvipdfmx, a4paper, 12pt, class=jsbook, crop=false]{standalone}
\usepackage{import}
\import{../}{common-preamble.sty}

\begin{document}
\section{第一可算空間}
\label{sec:first-countable-spaces}

\begin{definition}
	位相空間$ X $が\indexjj{だいいちかさんこうり}{第一可算公理}{first axiom of countability}を満たす,
	または\indexjj{だいいちかさん}{第一可算}{first countable}であるとは,
	任意の点$ x \in X $について可算な近傍基が存在することである.
\end{definition}

定義から第一可算空間の部分空間や第一可算空間の直和空間もまた第一可算空間であることがわかる.
第一可算性は連続写像では保たれない. 例えば, 非可算集合上の離散位相から補有限位相への恒等写像は
連続写像であり, 離散位相は第一可算であるが非可算集合上の補有限位相は第一可算でない.


\begin{proposition}
	第二可算空間は第一可算空間である.
\end{proposition}

\begin{proposition}
	距離空間は第一可算空間である.
\end{proposition}

\begin{proposition}
	コンパクト \Hausdorff 空間$ X $の任意の点が \Gdelta 集合であるとき,
	$ X $は第一可算空間である.
\end{proposition}

\end{document}
