\documentclass[uplatex, dvipdfmx, a4paper, 12pt, class=jsbook, crop=false]{standalone}
\usepackage{import}
\import{../}{common-preamble.sty}

\begin{document}
\section{距離空間}
\label{sec:metric-spaces}

\begin{definition}
	集合$ X $と写像$ d \colon X \times X \to \R $の組$ (X, d) $について次の4条件が成り立つとき,
	$ (X, d) $を\indexjj{きょりくうかん}{離化空間}{metric space}といい,
	$ d $を$ X $上の\indexjj{きょり}{距離}{metric}という.
	\begin{enumerate}
		\item 任意の$ x, y \in X $について$ d(x, y) \geq 0 $が成り立つ.
  		\item 任意の$ x, y \in X $について$ d(x, y) = 0 $と$ x = y $が同値である.
    	\item 任意の$ x, y \in X $について$ d(x, y) = d(y, x) $が成り立つ.
     	\item 任意の$ x, y, z \in X $について$ d(x, y) + d(y, z) \geq d(x, z) $が成り立つ.
	\end{enumerate}
	4つ目の条件の不等式は\indexjj{さんかくふとうしき}{三角不等式}{triangle inequality}とよばれる.
\end{definition}

\begin{definition}
	$ (X, d) $を距離空間とする. $ x \in X $と$ \varepsilon > 0 $に対して
	\[ B(x, \varepsilon) \defeq \setcomp{y \in X}{d(x, y) < \varepsilon} \]
	で定義される$ B(x, \varepsilon) $を$ x $の
	\indexjj{イプシロンきんぼう}{$\varepsilon$近傍}{epsilon neighborhood}という.
\end{definition}

\begin{proposition}
	$ f \colon X \to Y $を距離空間$ (X, d_X) $から距離空間$ (Y, d_Y) $への写像とする.
	このとき次はすべて同値である.
	\begin{enumerate}
		\item $ f $が連続である.
		\item 任意の点$ x $について次が成り立つ.
			\[\mbox{任意の} \varepsilon > 0 \mbox{について}
			\delta > 0 \mbox{が存在して} \mapset{f}{B(x, \delta)}
			\subset B(x, \mappt{f}{x}).\]
		\item $ $
	\end{enumerate}
\end{proposition}

\end{document}