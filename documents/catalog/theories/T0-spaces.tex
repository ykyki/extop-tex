\documentclass[uplatex, dvipdfmx, a4paper, 12pt, class=jsbook, crop=false]{standalone}
\usepackage{import}
\import{../}{common-preamble.sty}

\begin{document}
\section{\topT{0}空間, 対称空間}
\label{sec:T0-spaces}

\newcommand{\topleq}{\leq}
\newcommand{\topindis}{\fallingdotseq}

\begin{source}
	主に\cite[Chapter 16]{Schechter1997HAF}や\cite[Section 4.2]{GoubaultLarrecq2013NH}を参考にしている.
\end{source}

\begin{definition}
	位相空間$X$上の\indexjj{とくしゅかじゅんじょ}{特殊化順序}{specialization order}とは,
	\[ x \topleq y \defarw x \in \topcl \{y\} \]
	により定義される$X$上の二項関係である.
	この二項関係は集合$X$上の前順序となる.
	2点$x, y$について$x \leq y$かつ$y \leq x$が成り立つとき,
	$x, y$は\indexjj{いそうてきにくべつふのう}{位相的に区別不能}{topologically indistinguishable}であるといい,
	$x \topindis y$と書く.
\end{definition}

\begin{definition}
	位相空間$X$が\indexj{T0 くうかん}{\topT{0}空間}あるいは\indexj{\Kolmogorov くうかん}{\Kolmogorov 空間}であるとは,
	$X$の任意の2点$x, y$について$x \topindis y$ならば$x = y$となることである.
\end{definition}

\begin{definition}
	位相空間$X$が\indexjj{たいしょうくうかん}{対称空間}{symmetric space}であるとは, $X$の特殊化順序が対称的であることである.
\end{definition}

\begin{proposition}
	\label{05a202}
	\newcommand{\cM}{\mathcal{M}}
	位相空間$X$の2点$x$, $y$について以下の条件は同値である:
	\begin{enumerate}
		\item 特殊化順序について$x \leq y$が成り立つ.
		\item $X$の任意の閉集合$F$に対し, $y \in F$ならば$x \in F$となる.
		\item $X$の任意の開集合$G$に対し, $x \in G$ならば$y \in G$となる.
		\item $X$のある準開基$\cM$が存在して次が成り立つ: $\cM$の任意の元$M$に対し, $x \in M$ならば$y \in M$となる.
		\item 包含関係$\topnbd{x} \subset \topnbd{y}$が成り立つ.
	\end{enumerate}
	\qed
\end{proposition}

\begin{proposition}
	位相空間$X$について以下の条件は同値である:
	\begin{enumerate}
		\item $X$が\topT{0}空間である.
		\item $X$の特殊化順序が反対称的である.
		\item $X$の相異なる2点$x, y$に対し, ある開集合$G$が存在してどちらか1点のみを元に持つ.
		\item $X$の相異なる2点$x, y$に対し, ある閉集合$F$が存在してどちらか1点のみを元に持つ.
	\end{enumerate}
	\qed
\end{proposition}

\begin{proposition}
	位相空間$X$について以下の条件は同値である:
	\begin{enumerate}
		\item $X$が対称空間である.
		\item $X$の任意の点$x$に対し$\topbar{\{x\}} = \setcomp{y \in X}{x, y\text{は位相的に区別不能}}$となる.
		\item $X$の任意の点$x$と開集合$ G $に対し, $x \in G$ならば$\topbar{\{x\}} \subset G$である.
		\item $X$の任意の点$x$と閉集合$ F $に対し, $x \not\in F$ならば$\topbar{\{x\}} \cap F = \emptyset$である.
	\end{enumerate}
	\qed
\end{proposition}

\begin{example}[\Alexandroff 位相]
	\WIP.
\end{example}

\begin{example}[Upper Topology]
	\WIP.
\end{example}

\begin{example}[\Scott Topology]
	\WIP.
\end{example}

\begin{example}[\topT{0}空間ではあるが対称空間ではない空間]
	\WIP.
\end{example}

\begin{example}[対称空間ではあるが\topT{0}空間ではない空間]
	\WIP.
\end{example}

\begin{example}[\topT{0}空間でも対称空間でもない空間]
	\WIP.
\end{example}

位相空間$X$の部分空間$A$に対し, $A$上の特殊化順序は$X$上の特殊化順序を$A$に制限したものに等しい.
よって, \topT{0}空間や対称空間はそれぞれ部分空間に遺伝する.

\begin{proposition}
	\label{a1a202}
	積空間$X = \prod_{i \in I} X_i$上の2点$x = (x_i)$, $y = (y_i)$について以下の条件は同値である:
	\begin{enumerate}
		\item $X$上の特殊化順序について$x \topleq y$が成り立つ.
		\item 任意の$i \in I$に対し, 各$X_i$上の特殊化順序について$x_i \topleq y_i$が成り立つ.
	\end{enumerate}
\end{proposition}

\begin{proof}
	\cref{02b3a9}より$\topbar{\{y\}} = \prod_i \topbar{\{y_i\}}$が成り立つから.
\end{proof}

\begin{proof}
	\cref{05a202}より, 積空間の自然な準開基に注目すればよい.
\end{proof}

\begin{corollary}
	\topT{0}空間は積で保たれる.
	\qed
\end{corollary}

\begin{corollary}
	対称空間は積で保たれる.
	\qed
\end{corollary}

\begin{proposition}
	前正則空間は対称空間である.
\end{proposition}

\begin{proof}
	一般に,
	位相空間\(X\)上の2点\(x ,y\)について特殊化順序\(x \topleq y\)が成立しているとき,
	\cref{05a202}より\(x, y\)は近接する.
	よって, 前正則空間は対称空間である.
\end{proof}

\begin{proposition}
	\topT{1}空間は対称空間である.
	\qed
\end{proposition}

\begin{proposition}
	\topT{1}空間は\topT{0}空間である.
	\qed
\end{proposition}

\end{document}
