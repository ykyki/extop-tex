\documentclass[uplatex, dvipdfmx, a4paper, 12pt, class=jsbook, crop=false]{standalone}
\usepackage{import}
\import{../}{common-preamble.sty}

\begin{document}
\section{正規空間}
\label{sec:normal-spaces}

\begin{definition}
	位相空間$ X $が\indexjj{せいきくうかん}{正規空間}{normal space}であるとは, $ X $の互いに交わらない任意の閉集合$ F_1, F_2 $に対し, 互いに交わらないある開集合$ G_1, G_2 $が存在して$ F_1 \subset G_1, F_1 \subset G_2 $となることである.
\end{definition}

\begin{definition}
	位相空間$ X $が\indexj{T4 くうかん}{\topT{4}空間}であるとは,
	正規かつ\topT{1}空間であることである.
\end{definition}

\begin{proposition}
	\label{t400002}
	位相空間\( X \)が正規空間であることと,
	任意の閉集合\( F \)とそれを含む任意の開集合\( U \)について
	ある開集合\( V \)であって
	\( F \subset V, \topcl V \subset U \)
	を満たすものが存在することは同値である.
	\qed
\end{proposition}

\begin{theorem}[{\cite[Theorem 1]{HENNO200303}}]
	\label{t400001}
	位相空間\( X \)が正規であるための必要十分条件は,
	任意の閉集合\( F \)とそれを含む任意の開集合\( U \)
	に対して, ある開集合族\( \setcomp{U_n}{n \in \N} \)が存在して
	\( F \subset \bigcup_{n \in \N} U_n \)かつ
	\( \bigcup_{n \in \N} \topcl U_n \subset U \)が成り立つことである.
\end{theorem}

\begin{proof}
	\cref{t400002}より必要性は明らかなので十分性を示す.
	\( F, F' \)を互いに交わらない閉集合とする.
	このとき, 開集合族\( \setcomp{U_n}{n \in \N},
	\setcomp{V_n}{n \in \N} \)であって
	\[ F \subset \bigcup_{n \in \N} U_n,
	\ \ \bigcup_{n \in \N} \topcl U_n \subset X \setminus F' \]
	\[ F' \subset \bigcup_{n \in \N} V_n,
	\ \ \bigcup_{n \in \N} \topcl V_n \subset X \setminus F \]
	を満たすものが存在する.
	ここで, 各\( n \in \N \)について
	\( U'_n, V'_n \)をそれぞれ
	\[ U'_n \defeq U_n \setminus \lrparen{\bigcup_{m \leq n} \topcl V_m},
	\ \ V'_n \defeq V_n \setminus \lrparen{\bigcup_{m \leq n} \topcl U_m} \]
	と定める.
	このように定義される各\( U'_n, V'_n \)は明らかに開集合である.
	よって,
	\[ U'' \defeq \bigcup_{n \in \N} U'_n,
	\ \ V'' \defeq \bigcup_{n \in \N} V'_n \]
	と定めると, \( U'', V'' \)はともに\( X \)における開集合である.
	また, \( F \cap \lrparen{\bigcup_{n \in \N} \topcl V_n}= \emptyset,
	F' \cap \lrparen{\bigcup_{n \in \N} \topcl U_n} = \emptyset \)
	に注意すれば, \( F \subset U'', F' \subset V'' \)がわかる.
	\( U'' \cap V'' = \emptyset \)を示す.
	\[ \lrparen{\bigcup_{m\in \N} U'_m}
	\cap \lrparen{\bigcup_{n\in \N} V'_n}
	= \bigcup_{m \in \N} \bigcup_{n \in \N} \lrparen{U'_m \cap V'_n} \]
	が成り立つので, 任意の\( m, n \in N \)について
	\( U'_m \cap V'_n = \emptyset \)を示せばよい.
	ここで\( m \leq n \)として一般性を失わない.
	このとき,
	\[ U'_m \subset \bigcup_{n' \leq n} \topcl U_{n'} \]
	である.  よって, \( U'_m, V'_n \)の定義に注意すれば,
	\( U'_m \cap V'_n = \emptyset \)がわかる.
	以上により, \( X \)が正規空間であることが示された.
\end{proof}

\begin{theorem}
	正則空間\( X \)が \Lindelof であるならば正規空間である.
\end{theorem}

\begin{proof}
	\( F \)を閉集合とし, \( U \)を\( F \subset U \)なる開集合とする.
	\( X \)が \topT{3} 空間であることから,
	任意の点\( x \in F \)に対して,
	開集合\( U_x \)であって\( x \in U_x, \topcl U_x \subset U \)
	を満たすものが存在する. このとき,
	\[ F \subset \bigcup_{x \in F} U_x,
	\ \ \bigcup_{x \in F} \topcl U_x \subset U \]
	が成り立つ. \Lindelof 空間の閉部分空間も \Lindelof 空間なので,
	\( F \)の可算な部分集合\( G \)であって,
	\[  F \subset \bigcup_{x \in G} U_x,
	\ \ \bigcup_{x \in G} \topcl U_x \subset U \]
	を満たすものが存在する.
	したがって, 先の\cref{t400001}より\( X \)は \topT{4} 空間である.
\end{proof}

\begin{theorem}
	位相空間$ X $について以下の条件は同値である:
	\begin{enumerate}
		\item $ X $が正規である.
		\item $ X $の点有限な任意の開被覆に対し収縮が存在する.
		\item $ X $の互いに交わらない任意の閉集合$ F_1, F_2 $に対し, ある連続写像$ \phi \colon X \rightarrow \I $が存在して$ \phi[F_1] \subset \{0\}, \phi[F_2] \subset \{1\} $となる.
		\item $ X $の局所有限な任意の開被覆に対し, その開被覆に従属する1の分割が存在する.
	\end{enumerate}
\end{theorem}

\begin{lemma}
	\label{lem:Let X and Y are top.sp, Y is T2, f,g : X to Y are conti, then the set defined as {x in X | f(x) = g(x)} is closed.}
	$f, g \colon X \to Y $を位相空間$ X $から \topT{2} 位相空間$ Y $への連続写像とするとき, $ \setcomp{x \in X}{f(x) = g(x)} $は$ X $の閉集合である.
\end{lemma}

\begin{proof}
	$ U \defeq \setcomp{x \in X}{f(x) \neq g(x)} $とする. 任意の$ y \in U $に対して, $ f(y) \in V_1, g(y) \in V_2, V_1 \cap V_2 = \emptyset $を満たす開集合$ V_1, V_2 $が存在する. このとき, $ y \in f^{-1}(V_1) \cap g^{-1}(V_2) \subset U $である.
\end{proof}

\begin{theorem}
	\label{thm:Jone's Lemma}
	$ X $を \topT{4} 空間, $ D $を$ \cardinality{D} \geq \cardinality{\N} $なる$ X $の稠密部分集合, $ C $を$ X $の離散閉集合とする.このとき, $2^{\cardinality{C}} \leq 2^{\cardinality{D}} $が成り立つ.
\end{theorem}

\begin{proof}
	位相空間$ X $から$ Y $への連続写像全体の集合を$ C(X, Y) $と書くことにする. $ D $を$ X $の稠密部分集合とすると, 補題 \ref{lem:Let X and Y are top.sp, Y is T2, f,g : X to Y are conti, then the set defined as {x in X | f(x) = g(x)} is closed.} より$ f|_D = g|_D  \lrimp f =g $が成り立つので$ \cardinality{C(X, \I)} = \cardinality{C(D, \I)} \leq 2^{\cardinality{D}} $となる. また, $ C $を$ X $の離散閉集合とすると
	$$ \cardinality{C(C, \I)} =\begin{cases}
	2^{\cardinality{\N}} & \cardinality{C} \leq \cardinality{\N}\\
	2^{\cardinality{C}} & \cardinality{C} > \cardinality{\N}
	\end{cases} $$
	$ X $は \topT{4} であることから Tietze の拡張定理(\cref{t400004})より$ \cardinality{C(X, \I)} \geq \cardinality{C(C, \I)} $である. 以上より, $ 2^{\cardinality{C}} \leq 2^{\cardinality{D}} $となる.
\end{proof}

\begin{corollary}
	\label{coro:Corollary of Jone's Lamma}
	可分な \topT{4} 空間は, 非可算な離散閉集合を含まない.
\end{corollary}

\begin{proof}
	定理 \ref{thm:Jone's Lemma} から直ちに得られる.
\end{proof}

\begin{definition}
	\( X, Y \)を位相空間とし, \( A \)を\( X \)の部分集合とする.
	連続写像\( \morph{f}{A}{Y} \)に対して,
	連続写像\( \morph{g}{X}{Y} \)であって
	\( g|_A = f \)を満たすものを
	\( f \)の\( X \)上への拡張という.
\end{definition}

\begin{theorem}[\Tietze extension theorem, {\cite[定理~19.4]{Morita1981ja}}]
	\label{t400004}
	\( Y \)を距離空間\( \R \)の部分集合
	\( \intcc{a}{b}, \intoo{a}{b}, \R \)(\( a < b \))
	のいずれかとする.
	\( X \)を \topT{4} 空間とし, \( F \subset X \)を閉集合とする.
	\( F \)上の連続写像\( \morph{f}{F}{Y} \)に対して,
	\( f \)の\( X \)上への拡張\( \morph{g}{X}{Y} \)が存在する.
\end{theorem}

\begin{proof}
	\( Y = \intcc{a}{b} \)の場合を示す.
	\( c > 0 \)について\( Y = \intcc{-c}{c} \)の場合を示せばよい.
	\( \morph{h}{\intcc{a}{b}}{\intcc{-c}{c}} \)を同相写像とする.
	\( \morph{h \circ f}{F}{\intcc{-c}{c}} \)の拡張が
	\( \morph{g}{X}{\intcc{-c}{c}} \)であるとき,
	\( \morph{h^{-1}\circ g}{X}{\intcc{a}{b}} \)
	は\( f \)の拡張であることがわかる.
	よって, \( Y = \intcc{-c}{c} \)について拡張の存在を示す.
	\( f_1 \defeq f, c_1 \defeq c \)とおく.
	また, \( F \)の部分集合\( A_1, B_1 \)を次で定める:
	\[ A_1 \defeq \setcomp{x \in F}{\mappt{f}{c}
	\leq -\frac{c_1}{3}}, \
	B_1 \defeq \setcomp{x \in F}{\mappt{f}{x}
	\geq \frac{c_1}{3}} \]
	$ F $が$ X $の閉集合であることに注意すれば,
	\( A_1, B_1 \)は\( A_1 \cap B_1 = \emptyset \)
	を満たす\( X \)の閉集合である.
	よって, \Urysohn の補題より連続写像
	\( \morph{g_1}{X}{\intcc{-\frac{c_1}{3}}{\frac{c_1}{3}}} \)
	であって,
	\[ \mapset{g_1}{A_1} = \lrbrace{-\frac{c_1}{3}}, \
	\mapset{g_1}{B_1} = \lrbrace{\frac{c_1}{3}} \]
	を満たすものが存在する.
	このとき, \( x \in F \)に対して
	\( \mappt{f_2}{x} \defeq \mappt{f_1}{x} - \mappt{g_1}{x} \)
	と定めると次が成り立つ:
	\begin{align*}
	\mapset{f_1}{A_1}, \mapset{g_1}{A_1} & \subset \intcc{-c_1}{-\frac{c_1}{3}}, \\
	\mapset{f_1}{B_1}, \mapset{g_1}{B_1} & \subset \intcc{\frac{c_1}{3}}{c_1}, \\
	\mapset{f_1}{F \setminus \lrparen{A_1 \cup B_1}},
	\mapset{g_1}{F \setminus \lrparen{A_1 \cup B_1}} &
	\subset \intcc{-\frac{c_1}{3}}{\frac{c_1}{3}}
	\end{align*}
	よって, 任意の点\( x \in F \)について
	\( \abs{\mappt{f}{x}} \leq \frac{2}{3}c_1 \)が成り立つ.
	ゆえに, \( c_2 \defeq \frac{2}{3}c_1 \)とすれば,
	\( f_2 \)は\( F \)から\( \intcc{-c_2}{c_2} \)への連続写像である.
	自然数\( n \geq 2 \)と連続写像\( \morph{f_n}{F}{\intcc{-c_n}{c_n}} \)
	について, \( n = 1 \)のときと同様に
	\( \morph{g_n}{X}{\intcc{-\frac{c_n}{3}}{\frac{c_n}{3}}} \)
	をとることができる.
	このとき, \( x \in F \)について
	\( \mappt{f_{n+1}}{x}\defeq \mappt{f_n}{x} - \mappt{g_n}{x},
	c_{n+1} \defeq \frac{2}{3}c_n \)と定めると,
	\( \abs{\mappt{f_{n+1}}{x}} \leq c_{n+1} \)が成り立つ.
	以上から, 数学的帰納法により, 任意の\( n \geq 1 \)に対して
	\( \mappt{f_{n+1}}{x} \defeq \mappt{f_n}{x} - \mappt{g_n}{x},
	\abs{\mappt{f_{n+1}}{x}} \leq c_{n+1},
	\abs{\mappt{g_{n}}{x}} \leq c_n, c_{n+1} \defeq \frac{2}{3}c_n \)を満たす
	写像列\( \lrparen{f_n}, \lrparen{g_n} \)
	と実数列\( \lrparen{c_n} \)が定義できた.
	ここで, \( \sum_{n=1}^{\infty} \frac{1}{3}c_n
	= \frac{1}{3}\sum_{n=1}^{\infty} \lrparen{\frac{2}{3}}^{n-1}c = c \)
	に注意すれば, WeierstrassのM-判定法(\cref{t400004})より
	\( \lrparen{g_n} \)は一様収束し,
	\( \mappt{g}{x} \defeq \sum_{n=1}^{\infty}\mappt{g_n}{x} \)で定められる
	\( g \)は\( X \)上の連続写像である.
	また, \( \sum_{n=1}^{\infty} \frac{1}{3}c_n = c \)より
	任意の\( x \in X \)について
	\( \abs{\mappt{g}{x}} \leq c \)である.
	さらに, 任意の\( x \in F \)について
	\( \mappt{g_{n}}{x} = \mappt{f_n}{x} - \mappt{f_{n+1}}{x} \)より
	\[ \sum_{i=1}^{n} \mappt{g_i}{x} = \mappt{f_1}{x} - \mappt{f_{n+1}}{x} \]
	が成り立つ.
	\( \abs{\mappt{f_{n+1}}{x}} \leq c_{n+1}
	= \lrparen{\frac{2}{3}}^{n} c \)
	に注意すると,
	\[ \mappt{g}{x} = \sum_{n=1}^{\infty} \mappt{g_n}{x}
	= \lim_{n \to \infty} \lrparen{\mappt{f_1}{x}
	- \mappt{f_{n+1}}{x}} = \mappt{f_1}{x} = \mappt{f}{x} \]
	が成り立つ.
	よって, \( \morph{g}{X}{\intcc{-c}{c}} \)は\( f \)の\( X \)上への拡張である.

	次に\( Y = \intoo{a}{b} \)の場合を示す.
	これについても同様に,
	\( c > 0 \)について\( Y = \intoo{-c}{c} \)の場合のみ示せばよい.
	このとき, \( f \)を\( F \)から\( \intcc{-c}{c} \)
	への連続写像とみなせば上の証明より,
	\( X \)上への拡張\( \morph{g}{X}{\intcc{-c}{c}} \)がとれる.
	ここで, \( E \defeq \mapinvset{g}{\lrbrace{-c,c}} \)とすれば,
	\( E \)は\( X \)の閉集合であって\( E \cap F = \emptyset \)を満たす.
	よって, \Urysohn の補題より,
	連続写像\( \morph{h}{X}{\I} \)であって
	\[ \mapset{h}{E} = \lrbrace{0}, \ \mapset{h}{F} = \lrbrace{1} \]
	を満たすものが存在する.
	このとき, \( x \in X \)について
	\( \mappt{g'}{x} \defeq \mappt{g}{x}\mappt{h}{x} \)と定めると,
	\( g' \)は連続であって\( g'|_F = g|_F = f \)を満たす.
	また, もし\( \abs{\mappt{g'}{x}} = c \)
	であれば\( \abs{\mappt{g}{x}} = c \)より
	\( x \in E \)となり, \( \mappt{h}{x} = 0 \)から
	\( \mappt{g’}{x} = 0 \)となって矛盾する.
	よって, 任意の\( x \in X \)について
	\( \abs{\mappt{g'}{x}} < c \)である.
	ゆえに, \( \morph{g'}{X}{\intoo{-c}{c}} \)となって,
	これは\( f \)の拡張である.

	最後に\( Y = \R \)の場合を示す.
	\( \morph{h}{\R}{\intoo{-1}{1}} \)を
	同相写像として, \( \morph{h\circ f}{X}{\intoo{-1}{1}} \)
	の拡張\( g \)を取れば,
	\( \morph{h^{-1}\circ g}{X}{\R} \)が\( f \)の拡張となることがわかる.
\end{proof}


\begin{definition}
	位相空間$ X $が\indexjj{けいしょうてきせいき}{継承的正規}{hereditary normal}であるとは, $ X $の任意の部分空間が正規になることである.
\end{definition}

\begin{definition}
	位相空間$ X $が\indexj{T5 くうかん}{\topT{5}空間}であるとは, 継承的正規かつ\topT{0}となることである.
\end{definition}

\end{document}
