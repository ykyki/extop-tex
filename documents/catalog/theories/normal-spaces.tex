\documentclass[uplatex, dvipdfmx, a4paper, 12pt, class=jsbook, crop=false]{standalone}
\usepackage{import}
\import{../}{common-preamble.sty}

\begin{document}
\section{正規空間}
\label{sec:normal-spaces}

\begin{definition}
	位相空間$ X $が\indexjj{せいきくうかん}{正規空間}{normal space}であるとは, $ X $の互いに交わらない任意の閉集合$ F_1, F_2 $に対し, 互いに交わらないある開集合$ G_1, G_2 $が存在して$ F_1 \subset G_1, F_1 \subset G_2 $となることである.
\end{definition}

\begin{definition}
	位相空間$ X $が\indexj{T4 くうかん}{\topT{4}空間}であるとは, 正規かつ\topT{0}空間であることである. (\topT{1} ? 要追記)
\end{definition}

\begin{theorem}
	位相空間$ X $について以下の条件は同値である:
	\begin{enumerate}
		\item $ X $が正規である.
		\item $ X $の点有限な任意の開被覆に対し収縮が存在する.
		\item  $ X $の互いに交わらない任意の閉集合$ F_1, F_2 $に対し, ある連続写像$ \phi \colon X \rightarrow \I $が存在して$ \phi[F_1] \subset \{0\}, \phi[F_2] \subset \{1\} $となる.
		\item $ X $の局所有限な任意の開被覆に対し, その開被覆に従属する1の分割が存在する.
	\end{enumerate}
\end{theorem}

\begin{lemma}
	\label{lem:Let X and Y are top.sp, Y is T2, f,g : X to Y are conti, then the set defined as {x in X | f(x) = g(x)} is closed.}
	$f, g \colon X \to Y $を位相空間$ X $から \topT{2} 位相空間$ Y $への連続写像とするとき, $ \setcomp{x \in X}{f(x) = g(x)} $は$ X $の閉集合である.
\end{lemma}

\begin{proof}
	$ U \defeq \setcomp{x \in X}{f(x) \neq g(x)} $とする. 任意の$ y \in U $に対して, $ f(y) \in V_1, g(y) \in V_2, V_1 \cap V_2 = \emptyset $を満たす開集合$ V_1, V_2 $が存在する. このとき, $ y \in f^{-1}(V_1) \cap g^{-1}(V_2) \subset U $である.
\end{proof}

\begin{theorem}
	\label{thm:Jone's Lemma}
	$ X $を \topT{4} 空間, $ D $を$ \cardinality{D} \geq \cardinality{\N} $なる$ X $の稠密部分集合, $ C $を$ X $の離散閉集合とする.このとき, $2^{\cardinality{C}} \leq 2^{\cardinality{D}} $が成り立つ.
\end{theorem}

\begin{proof}
	位相空間$ X $から$ Y $への連続写像全体の集合を$ C(X, Y) $と書くことにする. $ D $を$ X $の稠密部分集合とすると, 補題 \ref{lem:Let X and Y are top.sp, Y is T2, f,g : X to Y are conti, then the set defined as {x in X | f(x) = g(x)} is closed.} より$ f|_D = g|_D  \lrimp f =g $が成り立つので$ \cardinality{C(X, \I)} = \cardinality{C(D, \I)} \leq 2^{\cardinality{D}} $となる. また, $ C $を$ X $の離散閉集合とすると
	$$ \cardinality{C(C, \I)} =\begin{cases}
	2^{\cardinality{\N}} & \cardinality{C} \leq \cardinality{\N}\\
	2^{\cardinality{C}} & \cardinality{C} > \cardinality{\N}
	\end{cases} $$
	$ X $は \topT{4} であることから Tietze の拡張定理より$ \cardinality{C(X, \I)} \geq \cardinality{C(C, \I)} $である. 以上より, $ 2^{\cardinality{C}} \leq 2^{\cardinality{D}} $となる.
\end{proof}

\begin{corollary}
	\label{coro:Corollary of Jone's Lamma}
	可分な \topT{4} 空間は, 非可算な離散閉集合を含まない.
\end{corollary}

\begin{proof}
	定理 \ref{thm:Jone's Lemma} から直ちに得られる.
\end{proof}

\begin{theorem}[\Tietze]
	\label{thm:Tietze's extension theorem}
	\topT{4} 空間$ X $の閉部分集合$ A $上の実連続関数$ f \colon A \to \R $は$ X $上の実連続関数$ g \colon X \to \R $に拡張される.
\end{theorem}

\begin{definition}
	位相空間$ X $が\indexjj{けいしょうてきせいき}{継承的正規}{hereditary normal}であるとは, $ X $の任意の部分空間が正規になることである.
\end{definition}

\begin{definition}
	位相空間$ X $が\indexj{T5 くうかん}{\topT{5}空間}であるとは, 継承的正規かつ\topT{0}となることである.
\end{definition}

\begin{theorem}[Urysohn's Lemma]
	\label{T400003}
	\topT{4} 空間\( (X, \mathcal{O}) \)の閉集合\( F \)と開集合\( U \)が\( F \subset U \)を満たすとき,
	連続写像\( \morph{f}{X}{\I} \)が存在して\( \mapset{f}{F} = \lrbrace{0},
	\mapset{f}{X \setminus U} = \lrbrace{1} \)を満たすものが存在する.
\end{theorem}

\begin{proof}
	\( X \)が \topT{4} 空間なので, \( W \in \mathcal{O} \)であって
	\( F \subset W, \topcl W \subset U \)を満たすものが存在する.
	任意の\( n \in \N \)について\( X \)の開集合族
	\( \mathscr{V}_n = \setcomp{V_{\frac{m}{2^n}}}{0\leq m \leq 2^n} \)
	であって,
	\[ 0 \leq m \leq m' \leq 2^n \ \Longrightarrow
	\topcl V_{\frac{m}{2^n}} \subset V_{\frac{m'}{2^n}} \ \ \ (\ast) \]
	を満たすものが存在することを数学的帰納法により示す.
	\( n = 0 \)のとき, \( V_0 \defeq W, V_1 \defeq U \)とすれば条件\( (\ast) \)を満たす.
	\( n = k \)のとき, \( (\ast) \)を満たす\( \mathscr{V}_k \)が存在すると仮定する.
	このとき, 任意の\( 0 \leq m \leq 2^k - 1 \)について
	\( \topcl V_{\frac{m}{2^k}} \subset V_{\frac{m+1}{2^k}} \)
	が成り立つ.
	よって,
	\[ \topcl V_{\frac{m}{2^k}} \subset V, \
	\topcl V \subset V_{\frac{m+1}{2^k}} \]
	を満たす開集合$ V $が存在するので
	\( \topcl V_{\frac{2m+1}{2^{k+1}}} \defeq V \)とする.
	また, \( \topcl V_{\frac{2m}{2^{k+1}}} = \topcl V_{\frac{m}{2^k}},
	V_{\frac{2(m+1)}{2^{k+1}}} = V_{\frac{m+1}{2^k}} \)
	に注意すれば,
	\[ \topcl V_{\frac{2m}{2^{k+1}}} \subset V_{\frac{2m+1}{2^{k+1}}}, \
	\topcl V_{\frac{2m+1}{2^{k+1}}} \subset V_{\frac{2(m+1)}{2^{k+1}}} \]
	となり, \( \mathscr{V}_{k+1} \)は条件\( (\ast) \)を満たす.
	以上より, 任意の\( n \in \N \)について条件\( (\ast) \)を満たす
	開集合族\( \mathscr{V}_n \)の存在が示された.

	次に, \( t \in \R \)について開集合\( W_t \)を次で定める:
	\[ W_t \defeq \begin{cases}
		\emptyset & t < 0 \\
		\bigcup \setcomp{V_{\frac{m}{2^n}}}{\frac{m}{2^n} \leq t} & 0 \leq t \leq 1 \\
		X & t > 1
	\end{cases}\]
	このとき, \( t < t' \)に対して\( \topcl W_t \subset W_{t'} \)である.
	\( t < 0 \)または\( t' > 1 \)のときこれは明らかである.
	\( 0 \leq t < t' \leq 1 \)のとき,
	ある\( m, n \in \N \)が存在して
	\( t < \frac{m}{2^n} < \frac{m+1}{2^n} < t' \)が成り立つ.
	このとき, \[ W_t \subset V_{\frac{m}{2^n}}, \
	\topcl V_{\frac{m}{2^n}} \subset V_{\frac{m+1}{2^n}} \subset W_{t'} \]
	なので\( \topcl W_t \subset W_{t'} \)である.

	次に写像\( \morph{f}{X}{\I} \)を次で定める:
	\[ f(x) \defeq \inf  \setcomp{t \in \R}{x \in W_t} \]
	\( W_t \)の定義から\( 0 \leq f(x) \leq 1 \)は明らかである.
	\( x \in F \)のとき, \( F \subset V_0 \)より\( f(x) = 0 \)である.
	また, \( x \in X \setminus U \)のとき, \( x \notin V_1 \)より\( f(x) = 1 \)である.
	よって, \( \mapset{f}{F} = \lrbrace{0}, \mapset{f}{X \setminus U} = \lrbrace{1} \)
	である.
	\( f \)の連続性を示せばよい.
	任意の\( x_0 \in X \)について\( t_0 \defeq f(x_0) \)とする.
	任意の\( \varepsilon > 0 \)について
	\( x \notin W_{t_0 - \frac{\varepsilon}{3}} \)なので,
	\( \topcl W_{t_0 - \frac{\varepsilon}{2}} \subset
	W_{t_0 - \frac{\varepsilon}{3}} \)に注意すれば
	\( x \notin \topcl W_{t_0 - \frac{\varepsilon}{2}} \)である.
	一方, \( x \in W_{t_0 + \frac{\varepsilon}{2}} \)なので
	\( W' \defeq W_{t_0 + \frac{\varepsilon}{2}} \cap
	\lrparen{X \setminus \topcl W_{t_0 - \frac{\varepsilon}{2}}} \)
	と定めれば\( W' \in \mathcal{N}^*(x_0) \)である.
	このとき, 任意の\( x \in W' \)について
	\( t_0 - \frac{\varepsilon}{2} \leq f(x) \leq t_0 + \frac{\varepsilon}{2} \)
	より\( |f(x) - t_0| \leq \frac{\varepsilon}{2} < \varepsilon \)である.
	よって, \( \mapset{f}{W'} \subset (t_0-\varepsilon, t_0 + \varepsilon) \)より
	\( f \)は\( x \)で連続である.
\end{proof}

\begin{corollary}
	\topT{4} 空間\( X \)の閉集合\( F \)
	と開集合\( U \)が\( F \subset U \)を満たすとする.
	実数\( a, b \in R \)が\( a < b \)を満たすとき,
	連続写像\( \morph{f}{X}{[a,b]} \)が存在して
	\( \mapset{f}{F} = \lrbrace{a}, \mapset{f}{X \setminus U} = \lrbrace{b} \)
	を満たすものが存在する.
	\qed
\end{corollary}

\end{document}
