\documentclass[uplatex, dvipdfmx, a4paper, 12pt, class=jsbook, crop=false]{standalone}
\usepackage{import}
\import{../}{common-preamble.sty}

\begin{document}
\section{正規空間}
\label{sec:normal-spaces}

\begin{definition}
	位相空間$ X $が\indexjj{せいきくうかん}{正規空間}{normal space}であるとは, $ X $の互いに交わらない任意の閉集合$ F_1, F_2 $に対し, 互いに交わらないある開集合$ G_1, G_2 $が存在して$ F_1 \subset G_1, F_1 \subset G_2 $となることである.
\end{definition}

\begin{definition}
	位相空間$ X $が\indexj{T4 くうかん}{\topT{4}空間}であるとは,
	正規かつ\topT{1}空間であることである.
\end{definition}

\begin{proposition}
	\label{t400002}
	位相空間\( X \)が正規空間であることと,
	任意の閉集合\( F \)とそれを含む任意の開集合\( U \)について
	ある開集合\( V \)であって
	\( F \subset V, \topcl V \subset U \)
	を満たすものが存在することは同値である.
\end{proposition}

\begin{theorem}[{\cite[Theorem 1]{HENNO200303}}]
	\label{t400001}
	位相空間\( X \)が正規であるための必要十分条件は,
	任意の閉集合\( F \)とそれを含む任意の開集合\( U \)
	に対して, ある開集合族\( \setcomp{U_n}{n \in \N} \)が存在して
	\( F \subset \bigcup_{n \in \N} U_n \)かつ
	\( \bigcup_{n \in \N} \topcl U_n \subset U \)が成り立つことである.
\end{theorem}

\begin{proof}
	必要性は明らかなので十分性を示す.
	\( F, F' \)を互いに交わらない閉集合とする.
	このとき, 開集合族\( \setcomp{U_n}{n \in \N},
	\setcomp{V_n}{n \in \N} \)であって
	\[ F \subset \bigcup_{n \in \N} U_n,
	\ \ \bigcup_{n \in \N} \topcl U_n \subset X \setminus F' \]
	\[ F' \subset \bigcup_{n \in \N} V_n,
	\ \ \bigcup_{n \in \N} \topcl V_n \subset X \setminus F \]
	を満たすものが存在する.
	ここで, 各\( n \in \N \)について
	\( U'_n, V'_n \)をそれぞれ
	\[ U'_n \defeq U_n \setminus \lrparen{\bigcup_{m \leq n} \topcl V_m},
	\ \ V'_n \defeq V_n \setminus \lrparen{\bigcup_{m \leq n} \topcl U_m} \]
	と定める.
	このように定義される各\( U'_n, V'_n \)は明らかに開集合である.
	よって,
	\[ U'' \defeq \bigcup_{n \in \N} U'_n,
	\ \ V'' \defeq \bigcup_{n \in \N} V'_n \]
	と定めると, \( U'', V'' \)はともに\( X \)における開集合である.
	また, \( F \cap \lrparen{\bigcup_{n \in \N} \topcl V_n}= \emptyset,
	F' \cap \lrparen{\bigcup_{n \in \N} \topcl U_n} = \emptyset \)
	に注意すれば, \( F \subset U'', F' \subset V'' \)がわかる.
	\( U'' \cap V'' = \emptyset \)を示す.
	\[ \lrparen{\bigcup_{m\in \N} U'_m}
	\cap \lrparen{\bigcup_{n\in \N} V'_n}
	= \bigcup_{m \in \N} \bigcup_{n \in \N} \lrparen{U'_m \cap V'_n} \]
	が成り立つので, 任意の\( m, n \in N \)について
	\( U'_m \cap V'_n = \emptyset \)を示せばよい.
	ここで\( m \leq n \)として一般性を失わない.
	このとき,
	\[ U'_m \subset \bigcup_{n' \leq n} \topcl U_{n'} \]
	である.  よって, \( U'_m, V'_n \)の定義に注意すれば,
	\( U'_m \cap V'_n = \emptyset \)がわかる.
	以上により, \( X \)が正規空間であることが示された.
\end{proof}

\begin{theorem}
	\topT{3} 空間\( X \)が \Lindelof であるならば \topT{4} 空間である.
\end{theorem}

\begin{proof}
	\( F \)を閉集合とし, \( U \)を\( F \subset U \)なる開集合とする.
	\( X \)が \topT{3} 空間であることから,
	任意の点\( x \in F \)に対して,
	開集合\( U_x \)であって\( x \in U_x, \topcl U_x \subset U \)
	を満たすものが存在する. このとき,
	\[ F \subset \bigcup_{x \in F} U_x,
	\ \ \bigcup_{x \in F} \topcl U_x \subset U \]
	が成り立つ. \Lindelof 空間の閉部分空間も \Lindelof 空間なので,
	\( F \)の可算な部分集合\( G \)であって,
	\[  F \subset \bigcup_{x \in G} U_x,
	\ \ \bigcup_{x \in G} \topcl U_x \subset U \]
	を満たすものが存在する.
	したがって, 先の\cref{t400001}より\( X \)は \topT{4} 空間である.
\end{proof}

\begin{theorem}
	位相空間$ X $について以下の条件は同値である:
	\begin{enumerate}
		\item $ X $が正規である.
		\item $ X $の点有限な任意の開被覆に対し収縮が存在する.
		\item  $ X $の互いに交わらない任意の閉集合$ F_1, F_2 $に対し, ある連続写像$ \phi \colon X \rightarrow \I $が存在して$ \phi[F_1] \subset \{0\}, \phi[F_2] \subset \{1\} $となる.
		\item $ X $の局所有限な任意の開被覆に対し, その開被覆に従属する1の分割が存在する.
	\end{enumerate}
\end{theorem}

\begin{lemma}
	\label{lem:Let X and Y are top.sp, Y is T2, f,g : X to Y are conti, then the set defined as {x in X | f(x) = g(x)} is closed.}
	$f, g \colon X \to Y $を位相空間$ X $から \topT{2} 位相空間$ Y $への連続写像とするとき, $ \setcomp{x \in X}{f(x) = g(x)} $は$ X $の閉集合である.
\end{lemma}

\begin{proof}
	$ U \defeq \setcomp{x \in X}{f(x) \neq g(x)} $とする. 任意の$ y \in U $に対して, $ f(y) \in V_1, g(y) \in V_2, V_1 \cap V_2 = \emptyset $を満たす開集合$ V_1, V_2 $が存在する. このとき, $ y \in f^{-1}(V_1) \cap g^{-1}(V_2) \subset U $である.
\end{proof}

\begin{theorem}
	\label{thm:Jone's Lemma}
	$ X $を \topT{4} 空間, $ D $を$ \cardinality{D} \geq \cardinality{\N} $なる$ X $の稠密部分集合, $ C $を$ X $の離散閉集合とする.このとき, $2^{\cardinality{C}} \leq 2^{\cardinality{D}} $が成り立つ.
\end{theorem}

\begin{proof}
	位相空間$ X $から$ Y $への連続写像全体の集合を$ C(X, Y) $と書くことにする. $ D $を$ X $の稠密部分集合とすると, 補題 \ref{lem:Let X and Y are top.sp, Y is T2, f,g : X to Y are conti, then the set defined as {x in X | f(x) = g(x)} is closed.} より$ f|_D = g|_D  \lrimp f =g $が成り立つので$ \cardinality{C(X, \I)} = \cardinality{C(D, \I)} \leq 2^{\cardinality{D}} $となる. また, $ C $を$ X $の離散閉集合とすると
	$$ \cardinality{C(C, \I)} =\begin{cases}
	2^{\cardinality{\N}} & \cardinality{C} \leq \cardinality{\N}\\
	2^{\cardinality{C}} & \cardinality{C} > \cardinality{\N}
	\end{cases} $$
	$ X $は \topT{4} であることから Tietze の拡張定理より$ \cardinality{C(X, \I)} \geq \cardinality{C(C, \I)} $である. 以上より, $ 2^{\cardinality{C}} \leq 2^{\cardinality{D}} $となる.
\end{proof}

\begin{corollary}
	\label{coro:Corollary of Jone's Lamma}
	可分な \topT{4} 空間は, 非可算な離散閉集合を含まない.
\end{corollary}

\begin{proof}
	定理 \ref{thm:Jone's Lemma} から直ちに得られる.
\end{proof}

\begin{theorem}[\Tietze]
	\label{thm:Tietze's extension theorem}
	\topT{4} 空間$ X $の閉部分集合$ A $上の実連続関数$ f \colon A \to \R $は$ X $上の実連続関数$ g \colon X \to \R $に拡張される.
\end{theorem}

\begin{definition}
	位相空間$ X $が\indexjj{けいしょうてきせいき}{継承的正規}{hereditary normal}であるとは, $ X $の任意の部分空間が正規になることである.
\end{definition}

\begin{definition}
	位相空間$ X $が\indexj{T5 くうかん}{\topT{5}空間}であるとは, 継承的正規かつ\topT{0}となることである.
\end{definition}

\end{document}
