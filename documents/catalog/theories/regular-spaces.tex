\documentclass[uplatex, dvipdfmx, a4paper, 12pt, class=jsbook, crop=false]{standalone}
\usepackage{import}
\import{../}{common-preamble.sty}

\begin{document}
\section{正則空間}
\label{sec:regular-spaces}

\begin{definition}
	位相空間$ X $が\indexjj{せいそくくうかん}{正則空間}{regular space}であるとは, $ X $の任意の点$ x $と閉集合$ F $に対し, $ x \not\in F $ならば互いに交わらないある開集合$ G, G^\prime $が存在して$ x \in G, F \subset G^\prime $となることである.
\end{definition}

\begin{definition}
	位相空間$ X $が\indexj{T3 くうかん}{\topT{3}空間}であるとは, 正則かつ\topT{0}空間であることである.
\end{definition}

\begin{proposition}
	\label{prop:A property equivalent to regularity}
	位相空間$ X $が正則であることは, 任意の点$ x \in X $とその任意の近傍$ U $に対して$ \topcl{V} \subset U $を満たす$ x $の近傍$ V $が存在することと同値である.
	\qed
\end{proposition}

\begin{proposition}
	完全正則空間$ X $は正則空間である.
\end{proposition}

\begin{proposition}
	正規かつ対称な位相空間$ X $は正則である.
\end{proposition}

\end{document}
