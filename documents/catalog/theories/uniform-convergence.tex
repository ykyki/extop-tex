\documentclass[uplatex, dvipdfmx, a4paper, 12pt, class=jsbook, crop=false]{standalone}
\usepackage{import}
\import{../}{common-preamble.sty}

\begin{document}
\section{一様収束}
\label{sec:uniform-convergence}

\begin{definition}
	\( X \)を位相空間とする.
	\( \morph{f}{X}{\R} \)と各\( n \in \N \)
	に対する\( \morph{f_n}{X}{\R} \)を連続写像とする.
	写像列\( \lrparen{f_n}_{n \in \N} \)が\( f \)に一様収束するとは,
	任意の\( \varepsilon > 0 \)に対して,
	ある\( N \in \N \)が存在して,
	次が成り立つことである:
	任意の点\( x \in X \)について
	\( n \in \N \)が$ n > N $を満たすならば
	\( \abs{\mappt{f}{x} - \mappt{f_n}{x}} < \varepsilon \)
	が成り立つ.

	また, 写像列$ \lrparen{f_n}_{n \in \N} $が$ f $に各点収束するとは,
	次が成り立つことである:
	任意の点\( x \in X \)について,
	点列\( \lrparen{f_n(x)}_{n \in \N} \)
	が\( \mappt{f}{x} \)に収束する.
\end{definition}

\begin{proposition}
	\label{uc00001}
	一様収束する写像列は各点収束する.
	\qed
\end{proposition}

\begin{proposition}
	\label{uc00002}
	\( X \)を位相空間とする.
	各\( n \in \N \)について\( \morph{f_n}{X}{\R} \)を連続写像とする.
	写像列\( \lrparen{f_n}_{n \in \N} \)
	が写像\( \morph{f}{X}{\R} \)に一様収束するとき\( f \)は連続である.
\end{proposition}

\begin{theorem}[WeierstrassのM-判定法, {\cite[定理~12.12]{Morita1981ja}}]
	\label{uc00003}
	\( X \)を位相空間とし,
	\( \lrparen{f_n}_{n \in \N} \)を\( X \)上の連続写像の列とする.
	ここで, 正の実数からなる列\( \lrparen{M_n}_{n \in \N} \)について
	\begin{enumerate}
		\item 任意の\( n \in \N \)と任意の\( x \in X \)
		に対して, \( \abs{\mappt{f_n}{x}} \leq M_n \)である.
		\item 無限級数\( \sum_{n \in \N} M_n \)が収束する.
	\end{enumerate}
	が成り立つとする.
	このとき, 任意の\( x \in X \)について
	\( \sum_{n \in \N} \mappt{f_n}{x} \)は収束し,
	写像\( \morph{f}{X}{\R} \)を\( \mappt{f}{x}
	\defeq \sum_{n \in \N} \mappt{f_n}{x} \)
	で定めると\( f \)は連続である.
\end{theorem}

\begin{proof}
	\( M \defeq \sum_{n \in \N} M_n \)とする.
	任意の\( n \in \N \)について
	写像\( \morph{g_n}{X}{\R} \)を
	\( \mappt{g_n}{x} \defeq \sum_{m=0}^{n} \mappt{f_m}{x} \)と定めると
	写像列\( \lrparen{g_n} \)は連続写像の列である.
	\( \lrparen{g_n} \)が\( f \)に一様収束することを示せば,
	\cref{uc00002}より\( f \)が連続であることがわかる.
	任意の\( n \in \N \)と任意の\( x \in X \)について
	\[ \abs{\mappt{f}{x} - \mappt{g_n}{x}}
	= \abs{\sum_{m \in \N, m>n} \mappt{f_m}{x}}
	\leq \sum_{m \in \N, m>n} M_m = M - \sum_{m=0}^n M_m \]
	が成り立つ.
	無限級数\( \sum_{n \in \N} M_n \)が収束することから,
	任意の正の実数\( \varepsilon > 0 \)に対して
	ある\( N \in \N \)が存在し, 任意の\( n \in \N \)について
	\[ n > N \ \Longrightarrow \ M - \sum_{m=0}^n M_m < \varepsilon \]
	が成り立つ.
	以上から, \( \lrparen{g_n} \)が\( f \)に一様収束することが示された.
\end{proof}

\end{document}
