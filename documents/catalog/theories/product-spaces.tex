\documentclass[uplatex, dvipdfmx, a4paper, 12pt, class=jsbook, crop=false]{standalone}
\usepackage{import}
\import{../}{common-preamble.sty}

\begin{document}
\section{積空間}
\label{sec:product-spaces}

\newcommand{\directedA}{\mathbf{A}} % 有向集合A
\newcommand{\proj}{\mathrm{pr}} % 射影

\begin{proposition}
	\label{02b3a9}
	位相空間の族$\setfamily{X_\lambda}{\lambda \in \Lambda}$
	の部分空間の族$\setfamily{A_\lambda}{\lambda \in \Lambda}$について,
	$ \topcl \left(\prod_\lambda A_\lambda\right) = \prod_\lambda \topcl A_\lambda $が成り立つ.
\end{proposition}

\begin{proof}
	($\subset$):
	点$x = (x_\lambda)$が部分集合$\prod A_\lambda$の閉包に属しているとする.
	添字$\mu \in \Lambda$をひとつ固定する.
	点$x_\mu \in X_\mu$の開近傍$U_\mu$を任意にとる.
	空間$\prod X_\lambda$の部分集合$U_\mu \times \prod_{\lambda \neq \mu} X_{\lambda}$は点$x$の開近傍であるから$\prod A_{\lambda}$と交わる.
	よって$U_\mu$と$A_\mu$も交わる. ゆえに$x_\mu \in \topcl A_\mu$である.

	($\supset$):
	点$x = (x_\lambda)$が部分集合$\prod \topcl A_\lambda$に属しているとする.
	$x$の開近傍を任意にとる.
	その近傍は, 有限個の相異なる$\lambda_1, \ldots, \lambda_n \in \Lambda$と各点$x_{\lambda_i}$の開近傍$U_{\lambda_i}$を用いて$U_{\lambda_1} \times \cdots \times U_{\lambda_n} \times \prod_{\lambda' \neq \lambda_1, \ldots, \lambda_n} X_{\lambda'}$と表せる開近傍を含む.
	後者が各$\lambda$で$A_\lambda$と交わっているため, 前者も$\prod A_\lambda$と交わる.
\end{proof}

\begin{proof}
	($\subset$):
	$\topcl \left(\prod A_\lambda\right)$の点$x = (x_\lambda)_\lambda$を任意に与える.
	$\prod A_\lambda$内のあるネット$(p_\alpha)_\alpha$が存在して$x$に収束する.
	このとき各$\lambda$について, 射影によって定まる$A_\lambda$内のネット$\proj_\lambda (p_\alpha)$が$x_\lambda$に収束しているから$x_\lambda \in \topcl A_\lambda$である.
	よって$x \in \prod \topcl A_\lambda$である.

	($\supset$):
	逆の包含を示す.
	点$x = (x_\lambda) \in \prod \topcl A_\lambda$を任意に与える.
	各$\lambda$について, $A_\lambda$内のネット$\setfamily{p^\lambda_\alpha}{\alpha \in \directedA_\lambda}$が存在して$x_\lambda$に収束する.
	このとき, 有向集合$\directedA \defeq \prod \directedA_\lambda$で添字づけられたネット$\setfamily{(p^\lambda_{\alpha_\lambda})_\lambda}{(\alpha_\lambda)_\lambda \in \directedA}$は$\prod A_\lambda$内にあり, また$x$に収束している.
	よって$x \in \topcl \prod A_\lambda$である.
\end{proof}

\begin{corollary}
	\label{02e5k8}
	$ \setfamily{X_\lambda}{\lambda \in \Lambda} $を位相空間の族とする.
	ここで, 各$ \lambda \in \Lambda $に対して$ A_\lambda \neq \emptyset $なる部分空間の族
	$ \setfamily{A_\lambda}{\lambda \in \Lambda} $について,
	$ \prod_\lambda A_\lambda $が閉集合であるための必要十分条件は
	任意の$ \lambda \in \Lambda $について$ A_\lambda $が$ X_\lambda $の閉集合となることである.
	\qed
\end{corollary}

\begin{proposition}
	\label{pr00001}
	\( \setfamily{X_\lambda}{\lambda \in \Lambda} \)を位相空間の列とし,
	\( X \defeq \prod_{\lambda \in \Lambda} X_\lambda \)をその積空間,
	\( \morph{p_\lambda}{X}{X_\lambda} \)を射影とする.
	各\( \lambda \in \Lambda \)について,
	\( X_\lambda \)の位相\( \mathcal{O}_\lambda \)が
	近傍系\( \mathscr{N}_\lambda = \setcomp{\mathscr{N}_\lambda(x_\lambda)}
	{x_\lambda \in X_\lambda} \)によって定まっているとする.
	ここで, 各\( x = (x_\lambda) \in X \)について\( \mathscr{N}(x) \)を
	\[ \mathscr{N}(x) \defeq \setcomp{\bigcap_{\lambda \in \Lambda'}
	\mapinvset{p_\lambda}{U_\lambda}}{\Lambda' \subset \Lambda, \Lambda' \mbox{は有限},
	\lambda \in \Lambda', U_\lambda \in \mathscr{N}_\lambda(x_\lambda)} \]
	と定める. このとき, \( \mathscr{N} \defeq
	\setcomp{\mathscr{N}(x)}{x \in X} \)で定まる\( X \)の
	部分集合族\( \mathscr{N} \)は近傍系を成す.
	また, \( \mathscr{N} \)から定まる位相は\( X \)に定まる積位相に一致する.
\end{proposition}

\begin{proof}
	\( \mathcal{O} \)を\( X \)に定まる積位相とする.
	各\( x=(x_\lambda)\in X \)について\( \mathscr{N}(x) \)が
	\( (X, \mathcal{O}) \)における\( x \)の近傍基であることを示せばよい.
	$ V $を$ x $の任意の開近傍とする.
	このとき, ある有限集合$ \Lambda_0 \subset \Lambda $と
	各$ \lambda \in \Lambda_0 $について
	$ X_\lambda $の開集合$ V_\lambda $が存在して,
	$ x \in \bigcap_{\lambda \in \Lambda_0}
	\mapinvset{p_\lambda}{V_\lambda} \subset V $
	が成り立つ.
	各$ \lambda \in \Lambda_0 $について
	$ x \in \mapinvset{p_\lambda}{V_\lambda} $なので,
	$ x_\lambda \in V_\lambda $となる.
	このとき, 開近傍$ U_\lambda \in \mathscr{N}_\lambda(x_\lambda) $が存在して
	$ U_\lambda \subset V_\lambda $が成り立つ.
	よって, $ U \defeq \bigcap_{\lambda \in \Lambda_0}
	\mapinvset{p_\lambda}{U_\lambda} $とすると
	$ U \in \mathscr{N} $かつ$ U \subset V $となるので
	$ \mathscr{N}(x) $は$ x $の近傍基である.
\end{proof}

\end{document}
