\documentclass[uplatex, dvipdfmx, a4paper, 12pt, class=jsbook, crop=false]{standalone}
\usepackage{import}
\import{../}{common-preamble.sty}

\begin{document}
\section{閉集合族}
\label{sec:closed-sets}

\begin{definition}
	$ X $を位相空間とする.
	開集合の補集合として表すことが出来る部分集合を$ X $の\indexjj{へいしゅうごう}{閉集合}{closed set}と呼ぶ.
\end{definition}

\begin{definition}
	$ X $を位相空間, $ A $を$ X $の部分集合とする.
	$ A $の\indexjj{へいほう}{閉包}{closure}とは, $ A $を含む閉集合全体の交叉のことである.
	この集合を$ \topcl_X A $あるいは単に$ \topcl A $や$ \topbar{A} $で表す.
\end{definition}

\begin{proposition}
	閉包の基本的性質.
\end{proposition}

\begin{proposition}
	閉包と収束の関係.
\end{proposition}

\begin{corollary}
	収束による閉集合の特徴づけ.
\end{corollary}

\begin{proposition}
	閉集合による位相の定義.
\end{proposition}

\begin{proposition}
	閉包作用素による位相の定義.
\end{proposition}

\end{document}
