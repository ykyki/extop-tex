\documentclass[uplatex, dvipdfmx, a4paper, 12pt, class=jsbook, crop=false]{standalone}
\usepackage{import}
\import{../}{common-preamble.sty}

\begin{document}
\section{集合と写像}
\label{sec:sets-and-maps}

この節では, 集合や写像に関する基本的な用語や記号をまとめる.

集合$X$の冪集合$\pow(X)$の部分集合のことを$X$の\indexj{ぶぶんしゅうごうぞく}{部分集合族}と呼ぶ.
より一般に, 集合からなる集合のことを\indexj{しゅうごうぞく}{集合族}という.
一方, ある集合で添字付けられた集合の集まりのことを\indexj{しゅうごうけい}{集合系}という.
集合$I$の各要素$i$に対して集合$X_i$が対応している集合系のことを
\(\setfamily{X_i}{i \in I}\)
のように表す.
これは集合$\setcomp{X_i}{i \in I}$とは区別される.

集合族または集合系のどちらか片方について定義された概念は,
自然に他方についても定義ができる.
例えば
「集合$X$の部分集合族$\mathscr{U}$が点有限であるとは, 任意の点$a \in X$に対し, $a \in U$となる$U \in \mathscr{U}$が有限個である」
と定義されるが, このとき集合系については
「集合$X$の部分集合系$\setfamily{U_i}{i \in I}$が点有限であるとは, 任意の点$a \in X$に対し, $a \in U_i$となる$i \in I$が有限個である」
と定義される.
そこで以降では, 集合族あるいは集合系の片方についてのみ定義を与え, 他方の定義は明記しないことがある.
また注意として, ある性質について, 集合族$\setcomp{U_i}{i \in I}$は定義を満たすが集合系$\setfamily{U_i}{i \in I}$は定義を満たさないことがある.
例えば上述の点有限性がその一例である.

\end{document}
