\documentclass[uplatex, dvipdfmx, a4paper, 12pt, class=jsbook, crop=false]{standalone}
\usepackage{import}
\import{../}{common-preamble.sty}

\begin{document}
\section{集合と写像}
\label{sec:sets-and-maps}

この節では, 集合や写像に関する基本的な用語や記号をまとめる.

集合$X$の冪集合$\pow(X)$の部分集合のことを$X$の\indexj{ぶぶんしゅうごうぞく}{部分集合族}と呼ぶ.
より一般に, 集合からなる集合のことを\indexj{しゅうごうぞく}{集合族}という.
一方, ある集合で添字付けられた集合の集まりのことを\indexj{そえじづけられたしゅうごうぞく}{添字付けられた集合族}
あるいは\indexj{しゅうごうけい}{集合系}という.

\end{document}
