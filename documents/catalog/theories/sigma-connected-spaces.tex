\documentclass[uplatex, dvipdfmx, a4paper, 12pt, class=jsbook, crop=false]{standalone}
\usepackage{import}
\import{../}{common-preamble.sty}

\begin{document}
\section{\sigmaConnected 空間}
\label{sec:sigma-connected-spaces}

\newcommand{\loclabel}[1]{\label{LocalLabel-\thepart-\thechapter-\thesection:#1}}
\newcommand{\locref}[1]{\ref{LocalLabel-\thepart-\thechapter-\thesection:#1}}
\newcommand{\longrel}[1]{\ #1\ }

\begin{definition}
	位相空間$ X $が\indexjj{しぐまれんけつ}{\sigmaConnected}{\ensuremath{\sigma}-connected}であるとは, $ X $を可算個の互いに交わらない閉集合の合併で表したとき常に, それら閉集合の中で少なくとも1つが$ X $と等しくなることである.
\end{definition}

\begin{theorem}[\Sierpinski の定理]
	\label{thm:Sierpinski_continuum}
	連続体$X$は\sigmaConnected である.
\end{theorem}

\begin{lemma}
	\loclabel{lemma:1}
	$ U $を位相空間$ X $の開集合, $ \setfamily{F_\lambda}{\lambda \in \Lambda} $を閉集合の族($ \Lambda \neq \emptyset $)とする. $ F_\lambda $の中で少なくともひとつはコンパクトであるとする(とくに$ X $がコンパクトであればこの条件は満たされる). また$ \bigcap_{\lambda} F_\lambda \subset U$となっているとする. このとき有限個の$ F_1, \ldots , F_n \in \{ F_\lambda\}$が存在して$ F_1 \cap \cdots \cap F_n \subset U $となる.
\end{lemma}
\begin{proof}
	コンパクトである$ F_0 \in \{F_\lambda\}$を取る. すると$ U \cap F_0 $は$ F_0 $の開集合, $ \setfamily{F_0 \cap F_\lambda}{\lambda \in \Lambda} $は$ F_0 $の閉集合の族であり, $ \bigcap_{\lambda} (F_0 \cap F_\lambda) \subset U \cap F_0 $となる. そこで最初から$ X $がコンパクトである条件のもとで証明すれば十分である.
	
	$ X $がコンパクトであるとする. いまコンパクト部分空間$ X \setminus U $が開集合の族$ (X \setminus F_\lambda) $で覆われているので, 有限個の$ F_1, \ldots, F_n $が存在して
	\[ X \setminus U \subset \bigcup_{i=1}^{n} X \setminus F_i \]
	となる. よって$ F_1 \cap \cdots \cap F_n \subset U $である.
\end{proof}

\begin{lemma}
	\loclabel{lemma:2}
	$ X $をコンパクト\Hausdorff 空間, $ x_0 $を$ X $の点とする. $ x_0 $を元に持つ連結成分を$ C $, $ x_0 $を元に持つ開閉集合全体の共通部分を$ Q $とおくと, $ C = Q $である.
\end{lemma}
\begin{proof}
	$ x_0 \in K $となる開閉集合$ K $全体の集合を$ \mathscr{K} $とおく. いま$ Q  = \bigcap \mathscr{K}$である.
	
	($ \subset $): $ K \in \mathscr{K} $を任意に取る. $ C \cap K $と$ C \setminus K $がともに$ C $の開閉集合であり, $ C \cap K \neq \emptyset $となるので, $ C $の連結性より$ C \subset K $である. よって$ C \subset Q $である.
	
	($ \supset $): $ Q $が連結であることを示せばよい. $ Q_1, Q_2 $を$ Q_1 \cap Q_2 = \emptyset, Q_1 \cup Q_2 = Q , x_0 \in Q_1$となる$ Q $の任意の閉集合とする. $ Q $が閉集合であるから, $ Q_1, Q_2 $は$ X $においても閉である. $ X $が正規空間であるから, 開集合$ U_1, U_2 \subset X $が存在して$ Q_1 \subset U_1, Q_2 \subset U_2, U_1 \cap U_2 = \emptyset $となる. $\bigcap \mathscr{K} = Q = U_1 \cup U_2 $に補題\locref{lemma:1}を適用して, 有限個の$ K_1 , \ldots, K_n \in \mathscr{K}$で$ \bigcap
	_{1}^{n} K_i \subset U_1 \cup U_2$となるものが存在する. ここで$ K_0 \defeq \bigcap
	_{1}^{n} K_i$とおけば$ K_0 \in \mathscr{K}, Q \subset K_0 \subset U_1 \cup U_2 $となる. すると
	\[ \topcl\left( U_1 \cap K_0 \right) \longrel{\subset} \left(\topcl U_1 \right) \cap K_0 \longrel{=} \left(\topcl U_1 \right) \cap (U_1 \cup U_2) \cap K_0 \longrel{=} U_1 \cap K_0\]
	となるので, $ U_1 \cap K_0 \in \mathscr{K} $である. よって
	\[ Q_2 \longrel{\subset} Q \longrel{\subset} U_1 \cap K_0 \longrel{\subset} U_1 \]
	となり, よって$ Q_2 = \emptyset $である. ゆえに$ Q $は連結である.
\end{proof}

\begin{lemma}
	\loclabel{lemma:3}
	$ A $を連続体$ X $の閉集合で$ \emptyset \neq A \neq X $となるものとする. 空間$ A $の任意の連結成分$ C $に対し, $ C \cap \topbry A \neq \emptyset $が成立する.
\end{lemma}
\begin{proof}
	点$ x_0 \in C $を固定する. $ x_0 \in K $となる$ A $の開閉集合$ K $全体の集合を$ \mathscr{K} $とおく. 補題\locref{lemma:2}より$ C = \bigcap \mathscr{K} $である. もし仮に$ C \cap \topbry A = \emptyset $であったとする.
	
	$ \topbry A \subset A $は$ A $の閉集合であるからコンパクトである. 仮定より
	\[ \bigcap_{K \in \mathscr{K}} K \cap \topbry A = C \cap \topbry A = \emptyset \]
	なので, コンパクト性と併せて$ K \cap \topbry A = \emptyset $となる$ K \in \mathscr{K} $の存在が分かる. $ K = U \cap A  $となる$ X $の開集合$ U $を取る. $ A = \topint A \cup \topbry A $と分割されているので, $ K = U \cap \topint A $である. よって$ K $は$ X $においても開閉集合である. $x_0 \in K $と$ X $の連結性より$ K = X $である. よって$ \topbry A = \emptyset $となり, ゆえに$ A $は開集合である. しかしこれは$ X $の連結性に矛盾する.
\end{proof}

\begin{lemma}
	\loclabel{lemma:4}
	連続体$ X $の互いに交わらない閉被覆$ \setfamily{F_i}{i \in \N} $が与えられており, $ F_i $の中でも少なくとも2つは非空であるとする. このときある連続体$ C \subset X $が存在して$ C \cap F_0 = \emptyset $となり, さらに$ C \cap F_1, C \cap F_2, \ldots $の中でも少なくとも2つは非空となる.
\end{lemma}
\begin{proof}
	$ F_n \neq \emptyset $となる$ n \neq 0 $を取る. $ X $が正規空間なので, 開集合$ U, V \subset X $で$ F_0 \subset U, F_n \subset V, U \cap V = \emptyset $となるものが存在する. 点$ x \in F_n $と連結成分$ x \in C \subset F_n $を取る. この$ C $が所望の条件を満たすは連続体になる: まず$ C \cap F_0 = \emptyset, C \cap F_n \neq \emptyset $が成立する. 補題\locref{lemma:3}より, $ C \cap \topbry\topcl V \neq \emptyset $なので, この集合の元$ y $が存在する.また$ V $の取り方から$ F_n \subset \topint \topcl V $である. そこで$ y \in F_k $となる$ k $を取れば$ k \neq n $かつ$ C \cap F_k \neq \emptyset $となる.
\end{proof}

\begin{proof}[定理\ref{thm:Sierpinski_continuum}]
	$ X $の互いに交わらない閉被覆$ \setfamily{F_i}{i \in \N} $が与えられているとする. もし仮に$ F_i \neq \emptyset $となる$ i $が少なくとも2つ存在したとする. 補題\locref{lemma:4}よりある連続体$ C_0 \subset X $が存在して$ C_0 \cap  F_0 = \emptyset$となり, さらに$ C \cap F_1, C \cap F_2, \ldots $の中でも少なくとも2つは非空となる. 補題\locref{lemma:4}を再び$ C_0 $と閉被覆$ \setfamily{C_0 \cap F_i}{i \in \N_{>0}} $について適用する. するとある連続体$ C_1 \subset C_0 $が存在して$ C_1 \cap  F_1 = \emptyset$となり, さらに$ C_1 \cap F_2, C \cap F_3, \ldots $の中でも少なくとも2つは非空となる. この操作を繰り返すことで, 連続体の下降列$ C_0 \supset C_1 \supset \cdots $で$ C_i \cap F_i = \emptyset, C_i \neq \emptyset $が任意の$ i \in \N $について成立するものが得られる. $ X $がコンパクトなので$ \bigcap_{i=0}^{\infty} C_i \neq \emptyset $である. しかし一方で$ \bigcap_{i=0}^{\infty} C_i = \left(\bigcap_{i=0}^{\infty} C_i \right) \cap \left( \bigcup_{i=0}^{\infty} F_i \right) = \emptyset $ である. よって矛盾する.
\end{proof}

% \begin{guide}
% 	\cite{Engelking_GeneralTopology_1989}(6.1.27).
% \end{guide}

% より手短に次のように証明できる.
% \begin{guide}
% 	\url{https://twitter.com/yamyam_topo/status/186432213246681088}
% \end{guide}
% \begin{proof}
% 	hoge.
% \end{proof}

\end{document}
