\documentclass[uplatex, dvipdfmx, a4paper, 12pt, class=jsbook, crop=false]{standalone}
\usepackage{import}
\import{../}{common-preamble.sty}

\begin{document}
\section{前正則空間, 推移空間}
\label{sec:preregular-spaces}

\newcommand{\topleq}{\leq}
\newcommand{\topnear}{\mathrel{\diamond}}
\newcommand{\convto}{\rightsquigarrow}
\newcommand{\Llim}{\mathop{\mathrm{Lim}}\nolimits}

\begin{definition}
	位相空間\(X\)の2点\(x, y\)が\indexj{きんせつする}{近接する}とは,
	\(x\)の近傍と\(y\)の近傍が必ず交わることである.
	このとき\(x \topnear y\)と書く.
\end{definition}

\begin{definition}
	位相空間\(X\)が\indexjj{ぜんせいそくくうかん}{前正則空間}{preregular space}であるとは,
	近接する任意の2点が位相的に区別不能であることである.
\end{definition}

\begin{definition}
	位相空間\(X\)が\indexj{すいいくうかん}{推移空間}であるとは,
	近接関係\(\topnear\)が推移的であることである.
\end{definition}

\begin{proposition}
	位相空間\(X\)について以下の条件は同値である:
	\begin{enumerate}
		\item \(X\)が前正則である.
		\item \(X\)の任意の2点\(x, y\)に対し, \(x \topnear y\)ならば\(x \leq y\)である.
	\end{enumerate}
\end{proposition}

\begin{proposition}
	前正則空間\(X\)は推移空間である.
\end{proposition}

\begin{proposition}
	前正則空間\(X\)は対称空間である.
\end{proposition}

\begin{example}[\Sierpinski 空間]
	\WIP.

	集合\(S \defeq \lrbrace{0, 1}\)上に,
	部分集合族\(\lrbrace{\emptyset, \lrbrace{1}, S}\)を開集合族として位相を定義する.
	このとき, 特殊化順序としても$0 \leq 1$が成り立つ.
	一方で$1 \leq 0$は成立しない.
	よって$S$は対称空間ではないが\topT{0}空間である.
	また$S$上の近接関係は自明であるため, 推移空間である.
	したがって$S$は前正則ではない.
\end{example}

\begin{example}[推移空間かつ対称空間であるが前正則ではない空間]
	\WIP.
\end{example}

\begin{proposition}
	正則空間は前正則空間である.
\end{proposition}

\begin{proof}
	前正則空間\(X\)の2点\(x, y\)が近接しているとする.
	このとき特殊化順序\(x \topleq y\)が成立することを示せば十分である.

	対偶を示す.
	\cref{05a202}より,
	ある閉集合\(F\)が存在して\(x \not\in F\)かつ\(y \in F\)を満たす.
	\(X\)が正則空間であるから,
	ある開集合\(G, G'\)が存在して\(x \in G\)かつ\(F \subset G'\)となる.
	よって\(x, y\)は近接していない.
\end{proof}

\end{document}

