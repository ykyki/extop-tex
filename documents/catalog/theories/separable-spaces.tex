\documentclass[uplatex, dvipdfmx, a4paper, 12pt, class=jsbook, crop=false]{standalone}
\usepackage{import}
\import{../}{common-preamble.sty}

\begin{document}
\section{可分空間}
\label{sec:separable-spaces}

\begin{definition}
	位相空間$ X $が\indexjj{かぶん}{可分}{separable}であるとは,
	可算な稠密部分集合が存在することである.
\end{definition}

\begin{theorem}[{\cite[The Hewitt-Marczewski-Pondiczery Theorem~2.3.15]{Engelking1989GT}}]
	\label{sep00001}
	\( \setfamily{X_\lambda}{\lambda \in \Lambda} \)を
	任意の\( \lambda \in \Lambda \)について
	\( \topdensity{X}_\lambda \leq \mathfrak{m}, \mathfrak{m} \geq \omega \)
	なる位相空間の列とし,
	\( \cardinality{\Lambda} \leq 2^{\mathfrak{m}} \)
	を満たすとする.
	このとき, 積空間\( X \defeq \prod_{\lambda \in \Lambda} X_\lambda \)
	について\( \topdensity{X} \leq \mathfrak{m} \)
	が成り立つ.
\end{theorem}

\begin{corollary}
	\label{sep00002}
	\( \setfamily{X_i}{i \in I} \)を可分空間の可算列とする.
	このとき, 積空間\( X \defeq \prod_{i \in \I} X_i \)
	は可分である.
\end{corollary}

\end{document}