\documentclass[uplatex, dvipdfmx, a4paper, 12pt, class=jsbook, crop=false]{standalone}
\usepackage{import}
\import{../}{common-preamble.sty}

\begin{document}
\section{完全正規空間}
\label{sec:perfectly-normal-spaces}

\begin{definition}
	位相空間$ X $が\indexjj{かんぜんせいき}{完全正規}{perfectly normal}であるとは,
	$ X $が正規空間であって,
	$ X $の任意の閉集合が\Gdelta 集合になることである.
\end{definition}

\begin{definition}
	位相空間$ X $が\indexj{T6 くうかん}{\topT{6}空間}であるとは,
	完全正規かつ\topT{0}となることである.
\end{definition}

\begin{definition}
	位相空間$ X $の部分集合$ A $が\indexjj{れいてんしゅうごう}{零点集合}{zero set}であるとは,
	ある連続写像$ \morph{f}{X}{\R} $が存在して$ A = \mapinvpt{f}{0} $となることである.
\end{definition}

\begin{proposition}
	位相空間\(X\)が\topT{6}であるための必要十分条件は,
	\(X\)が\topT{1}であって,
	\(X\)の任意の閉集合が零点集合であることである.
\end{proposition}

\begin{proposition}
	\label{prop:T_3 + hLind. implies T_6}
	位相空間$X$が \topT{3}かつ継承的 \Lindelof ならば \topT{6}である.
\end{proposition}
\begin{proof}
	$X$の任意の開集合が \Fsigma 集合になることを示せばよい.
	$U$を$X$の任意の開集合とする.
	任意の$x\in U$に対して, $x\in U_x\subset U$を満たす開集合$U_x$が存在し,
	${\cal U}=\{U_x\}_{x\in U}$は$U$の開被覆になる.
	$U$が \topT{3} であることから$\topcl {\cal V}< {\cal U}$を満たす開被覆${\cal V}$が存在する.
	また,
	$U$が \Lindelof であることから可算部分被覆${\cal W}\subset {\cal V}$が存在する.
	よって, $U=\bigcup {\cal W}=\bigcup \topcl {\cal W}$となる.
\end{proof}

\begin{proposition}
	距離化可能な位相空間は完全正規である.
\end{proposition}

\end{document}
