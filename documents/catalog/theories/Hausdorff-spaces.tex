\documentclass[uplatex, dvipdfmx, a4paper, 12pt, class=jsbook, crop=false]{standalone}
\usepackage{import}
\import{../}{common-preamble.sty}

\begin{document}
\section{\Hausdorff 空間}
\label{sec:Hausdorff-spaces}

\begin{definition}
	位相空間$ X $が\indexj{T2 くうかん}{\topT{2}空間}あるいは\indexj{Hausdorff くうかん}{Hausdorff 空間}であるとは, \topT{0}かつ前正則空間であることである.
\end{definition}

\begin{proposition}
	\label{T200001}
	位相空間$ X $について以下の条件は同値である:
	\begin{enumerate}
		\item $ X $が\topT{2}空間である.
		\item $ X $の相異なる2点$ x, y $に対し, 互いに交わらない開集合$ G, G^\prime $が存在して$ x \in G$かつ$ y \in G^\prime $となる. つまり$ x, y $が近接していない.
		\item $ X $上の任意のフィルターの極限点が高々1点である.
		\item $ W $を位相空間, $ W_0 $を$ W $の稠密部分集合, $ f_0 \colon W_0 \rightarrow X $を連続写像とする. このとき$ f_0 $の$ W $への連続な拡張$ f \colon W \rightarrow X$が高々1つである.
	\end{enumerate}
\end{proposition}

\begin{proposition}
	\label{T200002}
	\( \setfamily{X_\lambda}{\lambda \in \Lambda} \)を
	位相空間の列とする.
	このとき, 積空間\( X \defeq \prod_{\lambda \in \Lambda} X_\lambda \)
	が \Hausdorff 空間ならば任意の\( \lambda \in \Lambda \)
	について\( X_\lambda \)は \Hausdorff 空間である.
\end{proposition}

\begin{proof}
	ある\( \lambda_0 \)について\( X_{\lambda_0} \)
	が \Hausdorff でないと仮定して矛盾を導く.
	このとき, \( X_{\lambda_0} \)上のネット
	\( (p_i)_{i \in I} \)であって
	2点以上に収束するものが存在する.
	ここで, 各\( \lambda \neq \lambda_0 \)について
	\( X_\lambda \)の点を1つとり\( x_\lambda \)とする.
	さらに, 各\( i \in I \)について
	\( x_i = (x^i_\lambda)_{\lambda \in \Lambda} \in X \)
	を次で定める:
	\[ x^i_\lambda \defeq \begin{cases}
		x_\lambda & \lambda \neq \lambda_0 \\
		p_i & \lambda =\lambda_0
	\end{cases} \]
	このとき, \( X \)上のネット
	\( (x_i)_{i \in I} \)は2点以上に収束する.
	これは, \( X \)が \Hausdorff 空間であることに矛盾する.
\end{proof}

\end{document}
