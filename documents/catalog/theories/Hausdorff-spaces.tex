\documentclass[uplatex, dvipdfmx, a4paper, 12pt, class=jsbook, crop=false]{standalone}
\usepackage{import}
\import{../}{common-preamble.sty}

\begin{document}
\section{\Hausdorff 空間}
\label{sec:Hausdorff-spaces}

\begin{definition}
	位相空間$ X $が\indexj{T2 くうかん}{\topT{2}空間}あるいは\indexj{Hausdorff くうかん}{Hausdorff 空間}であるとは, \topT{0}かつ前正則空間であることである.
\end{definition}

\begin{proposition}
	位相空間$ X $について以下の条件は同値である:
	\begin{enumerate}
		\item $ X $が\topT{2}空間である.
		\item $ X $の相異なる2点$ x, y $に対し, 互いに交わらない開集合$ G, G^\prime $が存在して$ x \in G$かつ$ y \in G^\prime $となる. つまり$ x, y $が近接していない.
		\item $ X $上の任意のフィルターの極限点が高々1点である.
		\item $ W $を位相空間, $ W_0 $を$ W $の稠密部分集合, $ f_0 \colon W_0 \rightarrow X $を連続写像とする. このとき$ f_0 $の$ W $への連続な拡張$ f \colon W \rightarrow X$が高々1つである.
	\end{enumerate}
\end{proposition}

\end{document}
