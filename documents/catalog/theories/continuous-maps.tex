\documentclass[uplatex, dvipdfmx, a4paper, 12pt, class=jsbook, crop=false]{standalone}
\usepackage{import}
\import{../}{common-preamble.sty}

\begin{document}
\section{連続写像}
\label{sec:continuous-maps}

\begin{definition}
	$ X, Y $を位相空間, $ a $を$ X $の点, $ f \colon X \rightarrow Y $を写像とする.
	このとき$ f $が点$ a $で\indexjj{れんぞく}{連続}{continuous}であるとは, $ f(a) $の任意の近傍$ V $に対して$ a $のある近傍$ U $が存在して$ f[U] \subset V $となることである.
\end{definition}

\begin{proposition}
	連続性の言い換え.
\end{proposition}

\begin{definition}
	$ X, Y $を位相空間, $ f \colon X \rightarrow Y $を写像とする.
	このとき$ f $が\indexjj{れんぞく}{連続}{continuous}であるとは, $ Y $の開集合の逆像が常に$ X $の開集合となること.
\end{definition}

恒等写像, 合成写像の連続性.

\begin{proposition}
	連続性の同値な言い換え.
\end{proposition}

\end{document}
