\documentclass[uplatex, dvipdfmx, a4paper, 12pt, class=jsbook, crop=false]{standalone}
\usepackage{import}
\import{../}{common-preamble.sty}

\begin{document}
\section{連続写像}
\label{sec:continuous-maps}

\begin{definition}
	$ X, Y $を位相空間, $ a $を$ X $の点, $ \morph{f}{X}{Y} $を写像とする.
	このとき$ f $が点$ a $で\indexjj{れんぞく}{連続}{continuous}であるとは,
	$ f(a) $の任意の近傍$ V $に対して$ a $のある近傍$ U $が存在して$ \mapset{f}{U} \subset V $となることである.
	また, 任意の点$ a \in X $において$ f $が連続であるとき, $ f $は連続写像であるという.
\end{definition}

\begin{proposition}
	連続性の言い換え.
	\WIP.
\end{proposition}

恒等写像, 合成写像の連続性.
\WIP.

\end{document}
