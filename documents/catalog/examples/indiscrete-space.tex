\documentclass[uplatex, dvipdfmx, a4paper, 12pt, class=jsbook, crop=false]{standalone}
\usepackage{import}
\import{../}{common-preamble.sty}

\begin{document}
\section{密着空間}
\label{ex:indiscrete-space}

$ X $を集合とする.
$ X $上の\indexjj{みっちゃくいそう}{密着位相}{indiscrete topology}とは,
$ \emptyset $と$ X $のみが開集合として指定された位相のことである.
$ X $が一点集合の場合は離散位相と一致する.
ここでは, $ X $が2点以上からなる場合について述べる.
開集合が有限個なのでコンパクトかつ第二可算であるが,
空集合以外の開集合は全体集合のみなので\topT{0}でない.
また, 任意の部分空間も密着位相をもち, 任意の位相空間から$ X $への任意の写像は連続なので弧状連結かつ局所弧状連結である.

\end{document}
