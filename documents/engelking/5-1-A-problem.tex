\documentclass[uplatex, dvipdfmx, a4paper, 12pt, class=jsarticle, crop=false]{standalone}
\usepackage{import}
\import{./}{preamble.sty}

\begin{document}

\begin{problem}[5.1.A]\label{eng-5-1-A-problem}
    \begin{enumerate}
        \item \topT{1}位相空間\(X\)について,
        2つの開集合からなる開被覆が常にある開被覆で星型細分されるならば
        \(X\)は正規空間である.

        \item 正規空間の任意の有限開被覆はある有限開被覆で星型細分できることを直接示せ.

        \item 距離可能空間の任意の開被覆はある開被覆で星型細分できることを直接示せ.

        \item \Hausdorff 空間\(X\)がコンパクトであるための必要十分条件は,
        \(X\)の任意の開被覆が局所有限(あるいは点有限)な部分被覆をもつことである.
    \end{enumerate}
\end{problem}

\end{document}