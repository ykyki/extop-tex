\documentclass[uplatex, dvipdfmx, a4paper, 12pt, class=jsarticle, crop=false]{standalone}
\usepackage{import}
\import{./}{preamble.sty}

\begin{document}

\begin{proof}[1.1.A]\label{eng-1-1-A-proof}
	\(A \cap B \subset A\)
	と
	\(A \cap B \subset B\)
	より
	\(\topbar{A \cap B} \subset \topbar{A}\)
	と
	\(\topbar{A \cap B} \subset \topbar{B}\)
	が成り立つ.
	よって, \(\topbar{A \cap B}
	\subset \topbar{A} \cap \topbar{B}\)
	が成り立つ.

	\(\topbar{A} \subset \topbar{A \setminus B}
	\cup \topbar{B}\)
	より
	\(\topbar{A} \setminus \topbar{B}
	\subset \topbar{A \setminus B}\)
	が成り立つ.

	実数直線\(\R\)の部分集合として
	\(A \defeq\intoo{-1}{0},
	B \defeq \intoo{0}{1}\)
	をとれば,
	\(\topbar{A \cap B} = \emptyset,
	\topbar{A} \cap \topbar{B} =\lrbrace{0}\)
	となる.
	よって, \(\topbar{A \cap B} \subsetneq
	\topbar{A} \cap \topbar{B}\)
	である.

	実数直線\(\R\)の部分集合として
	\(A \defeq\intcc{0}{1},
	B \defeq \intoo{0}{1}\)
	をとれば,
	\(\topbar{A} \setminus
	\topbar{B} = \emptyset,
	\topbar{A \setminus B} =\lrbrace{0,1}\)
	となる.
	よって, \(\topbar{A} \setminus
	\topbar{B}\subsetneq
	\topbar{A \setminus B}\)
	である.
\end{proof}

\end{document}