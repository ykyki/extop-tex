\documentclass[uplatex, dvipdfmx, a4paper, 12pt, class=jsarticle, crop=false]{standalone}
\usepackage{import}
\import{./}{preamble.sty}

\begin{document}

\begin{proof}[1.1.A]\label{eng-1-1-A-proof2}
	\(A \cap B\)の触点は\(A\)
	の触点であるので
	\(\topbar{A \cap B} \subset A\)
	が成り立つ.
	\(B\)についても同様である.
	よって, \(\topbar{A \cap B} \subset
	\topbar{A} \cap \topbar{B}\)
	が成り立つ.

	\(x \in \topbar{A} \setminus
	\topbar{B}\)を任意にとる.
	このとき, \(x\)のある開近傍\(U\)が存在して
	\(U \cap B = \emptyset\)となる.
	\(x\)の任意の開近傍\(V\)をとる.
	このとき, \(U \cap V\)は\(x\)の開近傍なので
	\(A\)と交わるが, \(B\)とは交わらない.
	よって, \(\lrparen{U \cap V} \cap A
	\setminus B \neq \emptyset\)である.
	特に\(V\)は\(A \setminus B\)と交わるので
	\(x \in \topbar{A \setminus B}\)となる.

	具定例の構成は\cref{eng-1-1-A-proof}と同様である.
\end{proof}

\end{document}