\documentclass[uplatex, dvipdfmx, a4paper, 12pt, class=jsarticle, crop=false]{standalone}
\usepackage{import}
\import{./}{preamble.sty}

\begin{document}

\renewcommand{\labelenumi}{(\alph{enumi})}
\begin{proof}[1.5.E]\label{eng-1-5-E-proof}
	\begin{enumerate}
		\item (\( 1 \Longrightarrow 2\))を示す.
		ある写像の族
		\( \setcomp{\morph{f_\lambda}{X}{\R}}
		{\lambda \in \Lambda} \)
		が存在して, この写像の族によって
		\( X \) 上に生成される位相を
		\( \mathcal{O} \)とする.
		任意の点\( x \in X \)とその任意の開近傍\( U \)をとる.
		このとき, ある有限集合\( \Lambda' \subset \Lambda \)
		と各\( \lambda \in \Lambda' \)について
		\( \R \)の開集合\( U_\lambda \)が存在して
		\[ x \in \bigcap_{\lambda \in \Lambda'}
		\mapinvset{f_\lambda}{U_\lambda},
		\bigcap_{\lambda \in \Lambda'}
		\mapinvset{f_\lambda}{U_\lambda} \subset U \]
		が成り立つ.
		ここで, 各\( \lambda \in \Lambda' \)について
		写像\( \morph{g_\lambda}{X}{\R} \)と
		\( \R \)の開集合\( V_\lambda \)を
		\[ g_\lambda \semicolon y \mapsto
		\abs{\mappt{f_\lambda}{y} - \mappt{f_\lambda}{x}},
		V_\lambda \defeq \setcomp{y - f(x)}
		{y \in U_\lambda} \]
		と定める.
		任意の\( \lambda \in \Lambda' \)に対して,
		\( \mappt{g_\lambda}{x} = 0, 0 \in  V_\lambda \)
		が成り立つ.
		よって, ある\( \varepsilon > 0 \)が存在して
		\( \intoo{-\varepsilon}{\varepsilon}
		\subset \bigcap_{\lambda \in \Lambda'} V_\lambda\)
		かつ,
		\( x \in \mapinvset{g_\lambda}
		{\intoo{-\varepsilon}{\varepsilon}} \subset U\)
		となる.
		写像\( \morph{g}{X}{\R} \)を
		\[ g(y) \defeq \min \lrbrack{
		\max \setcomp{\mappt{g_\lambda}{y}}{\lambda \in \Lambda'}
		\cup \lrbrace{\varepsilon}} \]
		各\( g_\lambda \)は連続なので\( g \)も連続である.
		また, \( g(x) = 0 \)である.
		任意の\( y \in X \setminus \bigcap_{\lambda \in \Lambda'} V_\lambda \)
		に対しては, 
	\end{enumerate}
\end{proof}

\end{document}