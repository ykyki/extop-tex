\documentclass[uplatex, dvipdfmx, a4paper, 12pt, class=jsarticle, crop=false]{standalone}
\usepackage{import}
\import{./}{preamble.sty}

\begin{document}

\renewcommand{\labelenumi}{(\alph{enumi})}
\begin{proof}[1.5.E]\label{eng-1-5-E-proof}
	\begin{enumerate}
		\item (\( 1 \Longrightarrow 2\))を示す.
		ある写像の族
		\( \setcomp{\morph{f_\lambda}{X}{\R}}
		{\lambda \in \Lambda} \)
		が存在して, この写像の族によって
		\( X \) 上に生成される位相を
		\( \mathcal{O} \)とする.
		任意の点\( x \in X \)とその任意の開近傍\( U \)をとる.
		このとき, ある有限集合\( \Lambda' \subset \Lambda \)
		と各\( \lambda \in \Lambda' \)について
		\( \R \)の開集合\( U_\lambda \)が存在して
		\[ x \in \bigcap_{\lambda \in \Lambda'}
		\mapinvset{f_\lambda}{U_\lambda},
		\bigcap_{\lambda \in \Lambda'}
		\mapinvset{f_\lambda}{U_\lambda} \subset U \]
		が成り立つ.
		ここで, 各\( \lambda \in \Lambda' \)について
		写像\( \morph{g_\lambda}{X}{\R} \)と
		\( \R \)の開集合\( V_\lambda \)を
		\[ g_\lambda \semicolon y \mapsto
		\abs{\mappt{f_\lambda}{y} - \mappt{f_\lambda}{x}},
		V_\lambda \defeq \setcomp{y - f(x)}
		{y \in U_\lambda} \]
		と定める.
		任意の\( \lambda \in \Lambda' \)に対して,
		\( \mappt{g_\lambda}{x} = 0, 0 \in  V_\lambda \)
		が成り立つ.
		よって, ある\( \varepsilon > 0 \)が存在して
		\( \intoo{-\varepsilon}{\varepsilon}
		\subset \bigcap_{\lambda \in \Lambda'} V_\lambda\)
		かつ,
		\( x \in \mapinvset{g_\lambda}
		{\intoo{-\varepsilon}{\varepsilon}} \subset U\)
		となる.
		写像\( \morph{g}{X}{\R} \)を
		\[ g(y) \defeq \min \lrbrack{
		\max \setcomp{\mappt{g_\lambda}{y}}{\lambda \in \Lambda'},
		\lrbrace{\varepsilon}} \]
		と定める.
		各\( g_\lambda \)は連続なので\( g \)も連続である.
		また, \( g(x) = 0 \)である.
		任意の\( y \in X \setminus U \), \( \lambda \in \Lambda'\)について
		\( y \notin \mapinvset{g_\lambda}
		{\intoo{-\varepsilon}{\varepsilon}} \)より
		\( \mappt{g_\lambda}{y} \geq \varepsilon \)
		である. よって, \( \mappt{g}{y} = \varepsilon \)となる.
		写像\( \morph{h}{\R}{\I} \)を
		\( h \semicolon y \mapsto \frac{y}{\varepsilon} \)
		と定めると, \( \morph{h \compo g}{X}{\I} \)は連続であり,
		\( \mappt{h \compo g}{x} = 0,
		\mapset{h \compo g}{X \setminus U} = \lrbrace{\varepsilon} \)
		が成り立つので, \( X \)は \topT{3.5} 空間である.

		(\( 2 \Longrightarrow 1 \))を示す.
		\topT{3.5} 空間\( X \)の位相を\( \mathcal{O} \)とする.
		任意の点\( x \in X \)とその任意の開近傍\( U \)について,
		\( \mappt{f}{x} = 0, \mapset{f}{X \setminus U} = \lrbrace{1} \)
		を満たす連続写像\( \morph{f}{X}{\R} \)が存在する.
		この連続写像を\( f_{\lrparen{x,U}} \)と書くことにする.
		\( \setcomp{f_{\lrparen{x,U}}}{x \in X, U \in \mathcal{O}, x \in U} \)
		が生成する位相を\( \mathcal{O}' \)として
		\( \mathcal{O}' = \mathcal{O} \)を示す.
		生成する位相の最小性より\( \mathcal{O}' \subset \mathcal{O} \)である.
		任意に\( V \in \mathcal{O} \)をとる.
		任意の点\( y \in V \)に対して,
		\( \mappt{f_{\lrparen{y, V}}}{y} = 0,
		\mapset{f_{\lrparen{y, V}}}{X \setminus V} = \lrbrace{1} \)
		が成り立つ.
		よって, \( W \defeq \intco{0}{\frac{1}{2}} \)とすれば,
		\( V = \bigcup_{y \in V} \mapinvset{f_{\lrparen{y,V}}}{W} \)
		より, \( V \in \mathcal{O}' \)である.
		よって, \( \mathcal{O}' = \mathcal{O} \)である.

		\item (\( 1 \Longrightarrow 2 \))は明らかなので
		(\( 2 \Longrightarrow 1 \))を示す.
		各\( i = 1, 2 \)について\( \mathscr{A}_i \)
		が生成する位相を\( \mathcal{O}_{\mathscr{A}_i} \)とすると
		\( \mathscr{A}_1 = \mathscr{A}_2 \)より,
		\( \mathcal{O}_{\mathscr{A}_1} = \mathcal{O}_{\mathscr{A}_1} \)である.
		\( X \)が \topT{3.5} 空間なので
		\( \mathcal{O}_{\mathscr{A}_i} = \mathcal{O}_i \)より
		\( \mathcal{O}_1 = \mathcal{O}_2 \)となる.
	\end{enumerate}
\end{proof}

\end{document}