\documentclass[uplatex, dvipdfmx, a4paper, 12pt, class=jsarticle, crop=false]{standalone}
\usepackage{import}
\import{./}{preamble.sty}

\begin{document}

\renewcommand{\labelenumi}{(\alph{enumi})}
\renewcommand{\labelenumii}{(\arabic{enumii})}
\begin{problem}[1.5.E]\label{eng-1-5-E-problem}
	\begin{enumerate}
		\item \topT{1} 空間 \( X \) の位相 \( \mathcal{O} \)
		について次は同値である.
		\begin{enumerate}
			\item ある写像の族
			\( \setcomp{\morph{f_\lambda}{X}{\R}}
			{\lambda \in \Lambda} \)
			が存在して, この写像の族によって
			\( X \) 上に生成される位相が
			\( \mathcal{O} \) に一致する.
			\item \( X \) が \topT{3.5} 空間である.
		\end{enumerate}
		\item 集合 \( X \) 上に2つの位相
		\( \mathcal{O}_1, \mathcal{O}_2 \)
		が定まっており, 位相空間
		\( \lrparen{X, \mathcal{O}_1}, \lrparen{X, \mathcal{O}_2} \)
		はともに \topT{3.5} 空間であるとする.
		ここで, \( i = 1, 2 \) について,
		\( X \) 上の実数値関数からなる族 \( \mathscr{A}_i \) を
		\[ \mathscr{A}_i \defeq \setcomp{\morph{f}{X}{R}}
		{f\mbox{は位相}\mathcal{O}_i\mbox{に関して連続である.}} \]
		と定める.
		このとき次は同値である.
		\begin{enumerate}
			\item \( \mathcal{O}_1 = \mathcal{O}_2 \) である.
			\item \( \mathscr{A}_1 = \mathscr{A}_2 \) である.
		\end{enumerate}
	\end{enumerate}
\end{problem}

\end{document}