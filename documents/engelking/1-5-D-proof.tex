\documentclass[uplatex, dvipdfmx, a4paper, 12pt, class=jsarticle, crop=false]{standalone}
\usepackage{import}
\import{./}{preamble.sty}

\begin{document}

\renewcommand{\labelenumi}{(\alph{enumi})}
\begin{proof}[1.5.D]\label{eng-1-5-D-proof}
	\begin{enumerate}
		\item \topT{0} な正則空間\( X \)が \topT{1} 空間であることを示す.
		\( X \)の相異なる2点\( x, y \)をとる.
		\( X \)が \topT{0} 空間であることから,
		\( x \notin \topcl \lrbrace{y} \)
		または
		\( y \notin \topcl \lrbrace{x} \)
		が成り立つ.
		\( x \notin \topcl \lrbrace{y} \)
		が成り立つとして一般性を失わない.
		このとき, \( X \)が正則空間であることから,
		開集合\( U, V \)であって,
		\( x \in U, y \in V, U \cap V = \emptyset \)
		を満たすものが存在する.
		よって, \( X \)は \Hausdorff 空間であり,
		特に \topT{1} 空間である.

		\topT{0} な完全正則空間\( X \)
		が \topT{1} 空間であることも同様にして示される.

		\topT{0} な正規空間\( X \)であって,
		\topT{1} でない空間の存在を示す.
		集合\( X \defeq \lrbrace{0, 1} \)の位相
		\( \mathcal{O} \)を
		\( \mathcal{O} \defeq \lrbrace{\emptyset,
		\lrbrace{0}, X} \)
		で定義する.
		このとき, 位相空間\( \lrparen{X, \mathcal{O}} \)
		は \topT{0} であるが \topT{1}ではないことがわかる.
		また, \( \lrparen{X, \mathcal{O}} \)において
		互いに交わらない2つの閉集合\( F, G \)をとると,
		少なくとも一方は空集合なので正規空間であることがわかる.

		\item \( X \)の開集合基\( \mathscr{B} \)
		は\( \cardinality{\mathscr{B}}
		= \topweight{X} \)を満たすとする.
		ここで, 写像\( \morph{f}{X}{\pow(\mathscr{B})} \)
		を\( f(x) \defeq \setcomp{U \in \mathscr{B}}
		{x \in U} \)により定義する.
		\( X \)が \topT{0} であることから
		相異なる2点\( x, y \)をとると,
		ある\( U \in \mathscr{B} \)であって,
		\( x, y \)のどちらか一方のみを含むものが存在する.
		よって, \( \mappt{f}{x} \neq \mappt{f}{y} \)
		より\( f \)は単射である.
		ゆえに, \( \cardinality{X} \leq
		2^{\cardinality{{\topweight{X}}}} \)である.

		3点以上からなる集合\( X \)上の密着位相を考えると,
		これは \topT{0} でなく
		\( \cardinality{X} \leq
		2^{\cardinality{\topweight{X}}} \)
		を満たさない空間の例である.
		また, \( \topweight{X} \geq \omega \)であるような反例は
		次のように構成できる.
		集合\( X \defeq \omega \oplus 2^{2^\omega} \)に対して,
		\( \mathcal{C} \defeq \setcomp{F \subset \omega}
		{F\mbox{は有限集合}} \cup \lrbrace{X} \)
		とすれば,
		\( \mathcal{C} \)は\( X \)上の閉集合系をなす.
		この閉集合系から定まる位相空間は \topT{0} でなく,
		\( \topweight{X} = \omega \)
		を満たしている.

		次に \Hausdorff 空間\( X \)について,
		\( \cardinality{X} \leq
		2^{2^{\topdensity{X}}}\)
		を示す.
		\( A \subset X \)を\( X \)の稠密部分集合であって
		\( \cardinality{A} = \topdensity{X} \)
		を満たすものとする.
		各点\( x \in X \)について
		\( \mathcal{N}\lrparen{x} \)を\( x \)の開近傍基とする.
		ここで, 各\( x \in X \)について
		\( \mathcal{N}'\lrparen{x}
		\defeq \setcomp{U \cap A}{U \in \mathcal{N}\lrparen{x}} \)
		と定める.
		\( A \)が稠密なので, 任意の\( U \in \mathcal{N}'\lrparen{x} \)
		について\( \topbar{U \cap A} = \topbar{U} \)が成り立つ.
		\( X \)が \topT{2} であることから
		\[ \bigcap_{U \in \mathcal{N}'\lrparen{x}} \topbar{U}
		= \bigcap_{V \in \mathcal{N}\lrparen{x}} \topbar{V}
		= \lrbrace{x} \]
		が成り立つ.
		よって, \( x \neq y \)のとき,
		\( \mathcal{N}'\lrparen{x} \neq \mathcal{N}'\lrparen{y} \)
		である.
		ゆえに, 写像\( \morphto{f}{X}{\pow(\pow(A))}
		{x}{\mathcal{N}'\lrparen{x}} \)
		は単射である.
		したがって,
		\( \cardinality{X} \leq \cardinality{\pow(\pow(A))}
		\leq 2^{2^{\topdensity{X}}} \)
		が成り立つ.

		次に \Hausdorff 空間\( X \)について,
		\( \cardinality{X} \leq
		\topdensity{X}^{\topcharacter{X}} \)
		を示す.
		各点\( x \in X \)について
		\( \mathcal{N}\lrparen{x} \)を\( x \)の開近傍基であって
		\( \cardinality{\mathcal{N}\lrparen{x}} \leq \topcharacter{X} \)
		を満たすものとする.
		\( \topcharacter{X} < \omega \)のとき,
		\( X \)の位相は離散位相になるので
		不等式の成立は明らかである.
		よって, \( \topcharacter{X} \geq \omega \)
		の場合について示す.
		\( \mathscr{A} \subset \pow(A) \)を
		\( \mathscr{A} \defeq \setcomp{A' \in \pow(A)}
		{\cardinality{A'} \leq \topcharacter{X}} \)で定めると,
		\( \cardinality{\mathscr{A}}
		\leq \topdensity{X}^{\topcharacter{X}} \)
		が成り立つ.
		各\( U \in \mathcal{N}\lrparen{x} \)について
		\( U \cap A \neq \emptyset \)
		なので,1点\( p(x, U) \in U \cap A \)
		をとれる.
		ここで, \( A(x) \defeq \setcomp{p(x, U)}
		{U \in \mathcal{N}\lrparen{x}} \)
		とすると, \( A(x) \in \mathscr{A} \)である.
		また,
		\( \mathscr{A}(x) \defeq \setcomp{U \cap A(x)}
		{U \in \mathcal{N}\lrparen{x}} \)
		とすると,
		\( \mathscr{A}(x) \subset \mathscr{A} \)
		である.
		いま, \( \cardinality{\mathcal{N}\lrparen{x}} \leq \topcharacter{X} \)
		なので,
		\( \cardinality{\mathscr{A}(x)} \leq \topcharacter{X}\)
		が成り立つ.
		また, 任意の\( U \in \mathcal{N}\lrparen{x} \)に対して
		\( x \in \topbar{U \cap A(x)} \subset \topbar{U} \)
		が成り立つ.
		よって, \( X \)が \topT{2} であることから,
		\( x \neq y \)ならば
		\[ \bigcap_{U \in \mathscr{A}(x)} \topbar{U}
		= \bigcap_{V \in \mathcal{N}\lrparen{x}} \topbar{V}
		= \lrbrace{x} \]
		が成り立つ. ゆえに, \( x \neq y \)
		ならば
		\( \mathscr{A}(x) \neq \mathscr{A}(y) \)
		となる.
		ここで, \( \mathfrak{A} \defeq
		\setcomp{\mathscr{A}' \in \pow(\mathscr{A})}
		{\cardinality{\mathscr{A}'} \leq \topcharacter{X}} \)
		とすると,
		\( \cardinality{\mathfrak{A}}
		\leq \cardinality{\mathscr{A}}^{\topcharacter{X}} \)
		が成り立つ.
		よって,
		写像\( \morphto{f}{X}{\mathfrak{A}}{x}{\mathscr{A}(x)} \)
		の単射性を考慮すれば
		\( \cardinality{X} \leq \cardinality{\mathfrak{A}}
		\leq \cardinality{\mathscr{A}}^{\topcharacter{X}}
		\leq \topdensity{X}^{\topcharacter{X}} \)
		がわかる.

		最後に, \Hausdorff 空間について示した上の2つの不等式が
		\topT{2} でない \topT{1} 空間において成立するとは限らないことを示す.
		\( \cardinality{X} \leq
		2^{2^{\topdensity{X}}}\)が成り立たない例を構成する.
		集合\( X \defeq \pow(\pow(\pow(\N))) \)に補有限位相
		を定めた位相空間がそのような例である.
		この補有限位相空間\( X \)が \topT{2} でない
		\topT{1} 空間であることは明らかである.
		また, \( X \)の可算無限部分集合は稠密部分集合なので
		\( \topdensity{X} = \omega \)である.
		いま, \( \cardinality{X} = 2^{2^{2^{\omega}}} \)なので
		\( \cardinality{X} > 2^{2^{\topdensity{X}}} \)となる.
		2つ目の不等式\( \cardinality{X} \leq
		\topdensity{X}^{\topcharacter{X}} \)
		が成立しない例を構成する.
		catalogの具体例"\( \Q \)の無限点追加非コンパクト化"の方法を参考に,
		\( \omega \)に\( 2^{\omega_1} \)個の無限遠点を付け加えた空間が
		求める反例となっている. (WIP)
		\item \( A \subset X \)を\( X \)の稠密部分集合であって
		\( \cardinality{A} = \topdensity{X} \)
		を満たすものとする.
		また, \( \mathscr{B} \defeq \setcomp{\topint \topcl U}{U \in \pow(A)} \)
		とする.
		いま, \( A \)は稠密部分集合なので,
		任意の開集合\( U \subset X \)について
		\( \topbar{U \cap A} = \topbar{U} \supset U \)
		が成り立つ.
		\( X \)が \topT{3} 空間であることから,
		任意の\( x \in U \)について,
		\( x \in V_x, \topbar{V_x} \subset U \)
		を満たす\( V_x \)が存在する.
		よって,
		\( U = \bigcup_{x \in U} \topint \topcl V_x \)
		が成り立つので, \( \mathscr{B} \)は
		\( X \)の開基である.
		ゆえに,
		\( \topweight{X} \leq 2^{\topdensity{X}} \)
		が成り立つ.
		\topT{2} だが \topT{3} でない空間であって,
		不等式\( \topweight{X} \leq 2^{\topdensity{X}} \)
		を満たさない例は(WIP)
	\end{enumerate}
\end{proof}


\end{document}