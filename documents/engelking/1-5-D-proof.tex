\documentclass[uplatex, dvipdfmx, a4paper, 12pt, class=jsarticle, crop=false]{standalone}
\usepackage{import}
\import{./}{preamble.sty}

\begin{document}

\renewcommand{\labelenumi}{(\alph{enumi})}
\begin{proof}[1.5.D]\label{eng-1-5-D-proof}
	\begin{enumerate}
		\item \topT{0} な正則空間\( X \)が \topT{1} 空間であることを示す.
		\( X \)の相異なる2点\( x, y \)をとる.
		\( X \)が \topT{0} 空間であることから,
		\( x \notin \topcl \lrbrace{y} \)
		または
		\( y \notin \topcl \lrbrace{x} \)
		が成り立つ.
		\( x \notin \topcl \lrbrace{y} \)
		が成り立つとして一般性を失わない.
		このとき, \( X \)が正則空間であることから,
		開集合\( U, V \)であって,
		\( x \in U, y \in V, U \cap V = \emptyset \)
		を満たすものが存在する.
		よって, \( X \)は \Hausdorff 空間であり,
		特に \topT{1} 空間である.

		\topT{0} な完全正則空間\( X \)
		が \topT{1} 空間であることも同様にして示される.

		\topT{0} な正規空間\( X \)であって,
		\topT{1} でない空間の存在を示す.
		集合\( X \defeq \lrbrace{0, 1} \)の位相
		\( \mathcal{O} \)を
		\( \mathcal{O} \defeq \lrbrace{\emptyset,
		\lrbrace{0}, X} \)
		で定義する.
		このとき, 位相空間\( \lrparen{X, \mathcal{O}} \)
		は \topT{0} であるが \topT{1}ではないことがわかる.
		また, \( \lrparen{X, \mathcal{O}} \)において
		互いに交わらない2つの閉集合\( F, G \)をとると,
		少なくとも一方は空集合なので正規空間であることがわかる.

		\item \( X \)の開集合基\( \mathscr{B} \)
		は\( \cardinality{\mathscr{B}}
		= \topweight{X} \)を満たすとする.
		ここで, 写像\( \morph{f}{X}{\pow(\mathscr{B})} \)
		を\( f(x) \defeq \setcomp{U \in \mathscr{B}}
		{x \in U} \)により定義する.
		\( X \)が \topT{0} であることから
		相異なる2点\( x, y \)をとると,
		ある\( U \in \mathscr{B} \)であって,
		\( x, y \)のどちらか一方のみを含むものが存在する.
		よって, \( \mappt{f}{x} \neq \mappt{f}{y} \)
		より\( f \)は単射である.
		ゆえに, \( \cardinality{X} \leq
		2^{\cardinality{{\topweight{X}}}} \)である.

		3点以上からなる集合\( X \)上の密着位相を考えると,
		これは \topT{0} でなく
		\( \cardinality{X} \leq
		2^{\cardinality{\topweight{X}}} \)
		を満たさない空間の例である.

		後半の主張の証明は(WIP)

		\item (WIP)
	\end{enumerate}
\end{proof}


\end{document}