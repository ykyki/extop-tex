\documentclass[uplatex, dvipdfmx, a4paper, 12pt, class=jsarticle, crop=false]{standalone}
\usepackage{import}
\import{./}{preamble.sty}

\begin{document}

\begin{proof}[1.5.B]\label{eng-1-5-B-proof}

\end{proof}
	3, 2, 1の順に示す.
	初めに, 3を示す.
	\( A \)を\( X \)の有限部分集合とする.
	任意の点\( x \in X \)について
	\( A \setminus \lrbrace{x} \defeq \lrbrace{a_0, \ldots, a_n} \)
	とする(\( A \setminus \lrbrace{x} = \emptyset \)のとき
	\( x \notin A^\topd \)であることは明らかなので
	\( A \setminus \lrbrace{x} \neq \emptyset \)を仮定している).
	各\( i = 0, \ldots, n \)に対して,
	\( x \)の開近傍\( U_i \)であって
	\( a_i \notin U_i \)を満たすもが存在する.
	ここで, \( U \defeq \bigcap_{i=0}^n U_i \)と定めると,
	\( U \cap \lrparen{A \setminus \lrbrace{x}} = \emptyset \)となる.
	よって, \( A^\topd = \emptyset \)である.

	次に, 2の\( \topbar{\lrparen{A^\topd}} = A^\topd \)を示す.
	これは, \( \topbar{\lrparen{A^\topd}}
	\subset A^\topd \)を確認すればよい.
	任意に\( x \in \topbar{\lrparen{A}^\topd} \)をとると,
	\( x \)の任意の開近傍\( U \)について
	\( U \cap A^\topd \neq \emptyset \)が成り立つ.
	ここで, \( y \in U \cap A^\topd \)を1点とる.
	\( y = x \)のとき, \( x \in A^\topd \)は明らかである.
	\( y \neq x \)の場合についても
	\( x \in A^\topd \)を示す.
	今, \( X \)が \topT{1}空間なので,
	\( y \)のある開近傍\( V \)が存在して,
	\( x \notin V, V \subset U \)となる.
	また, \( y \in A^\topd \)より
	\( V \cap A \neq \emptyset \)が成り立つ.
	よって, \( x \notin V \)より
	\( V \cap \lrparen{A \setminus \lrbrace{x}}
	\neq \emptyset \)となる.
	また, \( V \subset U \)より
	\( U \cap \lrparen{A \setminus \lrbrace{x}}
	\neq \emptyset \)となる.
	ゆえに, \( x \in A^\topd \)より
	\( \topbar{\lrparen{A^\topd}}
	\subset A^\topd \)が示された.

	次に, 2の\( A^\topd = \lrparen{\topbar{A}}^\topd \)を示す.
	\( A^\topd \supset \lrparen{\topbar{A}}^\topd \)を示せばよい.
	任意に\( x \in \lrparen{\topbar{A}}^\topd \)をとる.
	\( x \)の任意の開近傍\( U \)について
	\( U \cap \lrparen{\topbar{A} \setminus \lrbrace{x}}
	\neq \emptyset \)である.
	\( y \in U \cap \lrparen{\topbar{A} \setminus \lrbrace{x}} \)
	をとる. このとき, \( y \neq x \)なので
	\( y \)のある開近傍\( V \)であって,
	\( x \notin V, V \subset U \)を満たすものが存在する.
	\( y \in \topbar{A} \)より
	\( V \cap A \neq \emptyset \)である.
	また, \( x \notin V \)より
	\( V \cap \lrparen{A \setminus \lrbrace{x}}
	\neq \emptyset \)となる.
	さらに, \( V \subset U \)より
	\( U \cap \lrparen{A \setminus \lrbrace{x}}
	\neq \emptyset \)となる.
	よって, \( x \in A^\topd \)より
	\( A^\topd \supset \lrparen{\topbar{A}}^\topd \)
	が示された.

	1を示す.
	任意の\( x \in \lrparen{A^\topd}^\topd \)について
	\( x \in \topbar{A^\topd \setminus \lrbrace{x}} \)である.
	2の結果から,
	\[ \topbar{A^\topd \setminus \lrbrace{x}}
	\subset \topbar{\lrparen{A^\topd}} = A^\topd \]
	なので, \( x \in A^\topd \)である.

	以下では, \( X \)が \topT{0} 空間の場合には
	上の主張が成り立つとは限らない例を示す.

	実数全体からなる集合\( \R \)に対して
	その部分集合族\( \mathcal{O} \)を次で定める:
	\[ \mathcal{O} \defeq \lrbrace{\emptyset}
	\cup \setcomp{U \in \pow(X)}
	{0 \in U} \]
	この\( \mathcal{O} \)
	が開集合系の公理を満たすことは容易に確かめられる.
	このようにして定まる位相空間
	\( (\R, \mathcal{O}) \)は \topT{0} 空間である.
	\( A \defeq \lrbrace{0} \)とすると,
	\[A^\topd = \R \setminus \lrbrace{0},
	\lrparen{\topbar{A}}^\topd = \R\]
	となり,
	\( A^\topd \neq \lrparen{\topbar{A}}^\topd \)
	である.
\end{document}