\documentclass[uplatex, dvipdfmx, a4paper, 12pt, class=jsarticle, crop=false]{standalone}
\usepackage{import}
\import{./}{preamble.sty}

\begin{document}

\begin{problem}[5.1.D]\label{eng-5-1-D-problem}
    コンパクト\Hausdorff 空間\(X\)と部分集合の\(M\)のペアであって,
    空間\(X_M\)が正規とならないものが存在する.
\end{problem}

\begin{hosoku}
    位相空間\(X\)とその部分集合\(M\)に対し,
    位相空間\(X_M\)を次のように定義する:
    集合\(X\)上の部分集合族
    \[
        \setcomp{U \cup K}
        {\text{
            \(U\)は位相空間\(X\)の開集合,
            \(K\)は\(X \setminus M\)の部分集合
        }}
    \]
    は開集合族の公理をみたし,
    これを位相に定まる位相空間を\(X_M\)と定める.
\end{hosoku}

\end{document}