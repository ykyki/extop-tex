\documentclass[uplatex, dvipdfmx, a4paper, 12pt, class=jsarticle, crop=false]{standalone}
\usepackage{import}
\import{./}{preamble.sty}

\begin{document}

\begin{proof}[1.5.A]\label{eng-1-5-A-proof}

\end{proof}
	\(X\)が \topT{0} 空間であれば, 相異なる2点\(x, y\)について
	ある開集合\(U\)が存在して, \(x \in U, y \notin U\)
	または\(x \notin U, y \in U\)が成り立つ.
	前者であれば\(x \notin \topbar{\lrbrace{x}}\),
	後者であれば\(y \notin \topbar{\lrbrace{y}}\)となる.
	\(x \in \topbar{\lrbrace{x}}, y \in \topbar{\lrbrace{y}}\)が常に成り立つことと
	合わせると\(\topbar{\lrbrace{x}} \neq \topbar{\lrbrace{y}}\)がわかる.

	逆に, \(\topbar{\lrbrace{x}}
	\neq \topbar{\lrbrace{y}}\)のとき,
	ある点\(z \in X\)であって\(\topbar{\lrbrace{x}}\)と
	\(\topbar{\lrbrace{y}}\)のどちらか一方のみに属するものが存在する.
	一般性を失うことなく\( z \in \topbar{\lrbrace{x}} \)を仮定してよい.
	このとき, \(z\)のある開近傍\(U\)であって
	\(x \in U, y \notin U\)を満たすものが存在する.
	よって, \(X\)は \topT{0} 空間である.
\end{document}