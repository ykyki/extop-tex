\documentclass[uplatex, dvipdfmx, a4paper, 12pt, class=jsarticle, crop=false]{standalone}
\usepackage{import}
\import{./}{preamble.sty}

\begin{document}

\begin{proof}[1.5.C]\label{eng-1-5-C-proof}
	\( \morph{f}{X}{X} \)を位相空間\( X \)
	のレトラクションとし,
	\( A \subset X \)をそのレトラクトとする.
	また, \( \morph{\mathrm{id}}{X}{X} \)
	を恒等写像とする.
	このとき, \( A = \setcomp{x \in X}
	{\mathrm{id}(x) = f(x)} \)が成り立つ.
	これは, 任意の\( a \in A \)について
	\( a = f(x) \)なる\( x \in X \)が存在し,
	\( f(a) = f \compo f(x) = f(x) = a\)
	が成り立つことからわかる.
	今, \( X \)は \Hausdorff 空間なので,
	命題catalog-h00001より\( A \)は閉集合である.
\end{proof}

\end{document}