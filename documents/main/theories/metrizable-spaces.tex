\documentclass[uplatex, dvipdfmx, a4paper, 12pt, class=jsbook, crop=false]{standalone}
\usepackage{import}
\import{../}{common-preamble.sty}

\begin{document}
\section{距離化可能空間}
\label{sec:metrizable-spaces}

\begin{definition}
	位相空間$ X $が\indexjj{ぎきょりかかのう}{擬距離化可能}{psudo-metrizable}であるとは, $ X $の位相がある擬距離から誘導されることである.
\end{definition}

\begin{definition}
	位相空間$ X $が\indexjj{ぎきょりかかのう}{距離化可能}{metrizable}であるとは, $ X $の位相がある距離から誘導されることである.
\end{definition}

\begin{theorem}[\Urysohn]
	\label{thm:Urysohn's metrization theorem}
	位相空間$X$が第二可算かつ\topT{4}ならば距離化可能である。
\end{theorem}

\end{document}
