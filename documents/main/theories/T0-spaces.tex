\documentclass[uplatex, dvipdfmx, a4paper, 12pt, class=jsbook, crop=false]{standalone}
\usepackage{import}
\import{../}{common-preamble.sty}

\begin{document}
\section{\topT{0}空間, 対称空間}
\label{sec:T0-spaces}

\newcommand{\topindis}{\fallingdotseq}

\begin{source}
	主に\cite[Chapter 16]{Schechter1997HAF}や\cite[Section 4.2]{GoubaultLarrecq2013NH}を参考にしている.
\end{source}

\begin{definition}
	位相空間$ X $上の\indexjj{とくしゅかじゅんじょ}{特殊化順序}{specialization order}とは,
	\[ x \leq y \defarw x \in \topcl \{y\} \]
	により定義される$ X $上の順序のことである.
	2点$ x, y  $について$ x \leq y $かつ$ y \leq x $が成り立つとき,
	$ x, y $は\indexjj{いそうてきにくべつふのう}{位相的に区別不能}{topologically indistinguishable}であるといい,
	$ x \topindis y $と書く.
\end{definition}

\begin{definition}
	位相空間$ X $が\indexj{T0 くうかん}{\topT{0}空間}あるいは\indexj{\Kolmogorov くうかん}{\Kolmogorov 空間}であるとは,
	$ X $の任意の2点$ x, y $について$ x \topindis y $ならば$ x = y $となることである.
\end{definition}

\begin{proposition}
	位相空間$ X $について以下の条件は同値である:
	\begin{enumerate}
		\item $ X $が\topT{0}空間である.
		\item $ X $の特殊化順序が反対称的である.
		\item $ X $の相異なる2点$ x, y $に対し, ある開集合$ G $が存在してどちらか1点のみを元に持つ.
	\end{enumerate}
	\qed
\end{proposition}

\begin{definition}
	位相空間$ X $が\indexjj{たいしょうくうかん}{対称空間}{symmetric space}であるとは, $ X $の特殊化順序が対称的であることである.
\end{definition}

\begin{proposition}
	位相空間$ X $について以下の条件は同値である:
	\begin{enumerate}
		\item $ X $が対称空間である.
		\item $ X $の任意の点$ x $と開集合$ G $に対し, $ x \in G$ならば$ \topbar{\{x\}} \subset G $である.
		\item $ X $の任意の点$ x $と閉集合$ F $に対し, $ x \not\in F $ならば$ \topbar{\{x\}} \cap F = \emptyset $である.
		\item $ X $の任意の点$ x $に対し$ \topbar{\{x\}} = \setcomp{y \in X}{x, y\text{は位相的に区別不能}} $となる.
	\end{enumerate}
	\qed
\end{proposition}

\end{document}
