\documentclass[uplatex, dvipdfmx, a4paper, 12pt, class=jsbook, crop=false]{standalone}
\usepackage{import}
\import{../}{common-preamble.sty}
% \setlayout{product}
% \setlayout{develop}
\setlayout{review}

\usepackage{lipsum}

\begin{document}
\section{出力テスト}
\label{sec:test-A}

出力テスト用のドキュメントです.
\cref{sec:test-A}を参照せよ.
\cref{sec:test-B}を参照せよ.
\cref{sec:test-C}を参照せよ.
\cref{sec:test-D}を参照せよ.
\cref{sec:test-A,sec:test-B,sec:test-D}を参照せよ. %コンマの間に空白を入れると上手く表示されないことがある
\cite[Theorem~5.6.1]{Engelking1995TD}を参照せよ.
\cite[Theorem~5.6.1, Theorem~5.8.3, Proposition~5.9.10]{Engelking1995TD}を参照せよ.
\cite[132--134]{Morita1981ja}を参照せよ. % cf. texdoc biblatex -> 3.14.3 Page Numbers in Citations
\cite{KodamaNagami1974ja}を参照せよ.

\begin{itemize}
	\item 例えば\cref{def:test-def-A}を見よ.
	\item 例えば\cref{def:test-def-B}を見よ.
	\item 例えば\cref{thm:test-thm-A}を見よ.
	\item 例えば\cref{prop:test-prop-A}を見よ.
	\item 例えば\cref{pf:test-proof-A}を見よ.
	\item 例えば\cref{pf:test-proof-B}を見よ.
	\item 例えば\cref{cor:test-corollary-A}を見よ.
	\item 例えば\cref{prob:test-problem-A}を見よ.
	\item 例えば\cite{KodamaNagami1974ja}を見よ.
	\item 例えば\cite{Morita1981ja}を見よ.
\end{itemize}

像, 逆像に関する記号として
$f(x)$,
$\mappt{f}{x}$,
$g(x^3 - 2x + 1)$,
$\mappt{g}{x^3 - 2x + 1}$,
$\sin(\frac{n}{n+1})$,
$\mappt{\sin}{\frac{n}{n+1}}$,
$\mapinvpt{f}{x}$,
$\mapinvpt{p_X}{x_1}$,
$\mapset{f}{A}$,
$\mapset{f}{A \cap B}$,
$\mapset{f}{\R}$,
$\mapset{f}{\topcl_X D}$,
$\mapset{f_X}{\topcl D}$,
$\mapset{h}{\topcl_X D}$,
$\mapset{gf}{H}$,
$\mapset{f}{\prod_i A_i}$,
$\mapinvset{f}{A}$,
$\mapinvset{f}{A \cap B}$,
$\mapinvset{f}{\R}$,
$\mapinvset{f}{\topcl_X D}$,
$\mapinvset{f_X}{\topcl D}$,
$\mapinvset{h}{\topcl_X D}$,
$\mapinvset{(gf)}{H}$,
$\mapinvset{f}{\prod_i A_i}$,
% $\imageto{f}(A)$,
% $\imageot{f}(A)$
など.

\begin{definition}
	\label{def:test-def-A}
	\lipsum[1][1-4]
\end{definition}

\begin{definition}[テスト定義B]
	\label{def:test-def-B}
	\lipsum[1][2]
\end{definition}

\begin{theorem}[テスト定理A]
	\label{thm:test-thm-A}
	\lipsum[1][2]
\end{theorem}

\begin{corollary}
	\label{cor:test-corollary-A}
	\lipsum[1][2]
	\qed
\end{corollary}

\begin{corollary}
	\label{cor:test-corollary-B}
	\lipsum[1][2]
\end{corollary}

\begin{proposition}
	\label{prop:test-prop-A}
	\lipsum[1][2]
\end{proposition}

\begin{proof}
	\label{pf:test-proof-A}
	1つ目の証明.
\end{proof}

\begin{proof}
	\label{pf:test-proof-B}
	2つ目の証明.
\end{proof}

\begin{problem}
	\label{prob:test-problem-A}
	\lipsum[1][2]
\end{problem}

\section{出力テストB}
\label{sec:test-B}
\lipsum[2-3]

\section{出力テストC}
\label{sec:test-C}
\lipsum[4][1]

\section{出力テストD}
\label{sec:test-D}
\lipsum[4][1-3]

\end{document}
